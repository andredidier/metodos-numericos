\section {Introdução}

\begin{frame}
\frametitle{Introdução}
Todas as questões foram obtidas da 3ª edição do livro ``Métodos Numéricos'' de
José Dias dos Santos e Zanoni Carvalho da Silva.
\end{frame}

\section{Noções de Aritmética de Máquina}
\subsection[Ponto flutuante]{Ponto flutuante, aritmética e arredondamento}

\subsubsection{Questão 1.6}

\begin{frame}
\frametitle{Questão 1.6}
\framesubtitle{Representação numérica, aritmética de ponto flutuante,
arredondamento}

Considere a máquina $\machine{10}{5}{-9}{9}$. Nela, verifique se 
$(a+b)+c = a + (b+c)$, onde
\begin{align*}
a &= 32.424\\
b &= 4.2131\\
c &= 0.000382
\end{align*} 

\end{frame}

\begin{frame}
\frametitle{Questão 1.6}
\framesubtitle{$\machine{10}{5}{-9}{9}$}

Normalizando os números, temos:
\begin{align*}
a = 32.424 &= \machinenumber{3.2424}{1}\\
b = 4.2131 &= \machinenumber{4.2131}{0}\\
c = 0.000382 &= \machinenumber{3.8200}{-4}
\end{align*} 


\end{frame}

\begin{frame}
\frametitle{Questão 1.6}
\framesubtitle{$\machine{10}{5}{-9}{9}$, 
$a = \machinenumber{3.4240}{1}$, 
$b = \machinenumber{4.2131}{0}$,
$c = \machinenumber{3.8200}{-4}$}

Fazendo $(a+b)+c$, temos:
{\tiny
\begin{align*}
& (\machinenumber{3.2424}{1} + \machinenumber{4.2131}{0}) +
\machinenumber{3.8200}{-4} \\
%
= &(\machinenumber{3.2424}{1} + \machinenumber{0.42131}{1}) +
\machinenumber{3.8200}{-4}\\
%
= &\machinenumber{3.6637}{1} +\machinenumber{3.8200}{-4}\\
%
= &\machinenumber{3.6637}{1} +\machinenumber{0.0000382}{1}\\
%
= &\machinenumber{3.6637}{1}\\
\end{align*}
}

E fazendo $a + (b+c)$, temos:
{\tiny
\begin{align*}
& \machinenumber{3.2424}{1} + 
  (\machinenumber{4.2131}{0} + \machinenumber{3.8200}{-4})\\
= &\machinenumber{3.2424}{1} + 
  (\machinenumber{4.2131}{0} + \machinenumber{0.000382}{0})\\
= &\machinenumber{3.2424}{1} + \alert<2->{\machinenumber{4.2135}{0}}
\onslide<2->{\text{ [arredondamento]}}\\
= &\machinenumber{3.2424}{1} + \machinenumber{0.42135}{1}\\
= &\alert<2->{\machinenumber{3.6638}{1}}\onslide<2->{\text{ [arredondamento]}}
\end{align*}
}
\onslide<2->{Os resultados são diferentes.}

\end{frame}

\subsubsection{Questão 1.7}

\begin{frame}
\frametitle{Questão 1.7}
\framesubtitle{Representação numérica, aritmética de ponto flutuante}

Considere um computador hipotético que trabalha na base 10, com 5 dígitos no
significando e 2 dígitos no expoente, denotado por $\machine{10}{5}{-99}{99}$.
Nele calcue o valor de:
\[
S = \sum_{n=0}^{4} \frac{1}{7^n}
\]
de duas formas: (i) da maior parcela para a menor e (ii) da menor parcela para a
maior.

O que dizer\footnote{Alguma propriedade dos números reais não foi verificada?
Dos dois resultados, qual o mais próximo do verdadeiro? Etc.} diante dos
resultados dos itens (i) e (ii)?

\end{frame}

\begin{frame}
\frametitle{Questão 1.7}
\framesubtitle{$\machine{10}{5}{-99}{99}$}

Em (i), calculamos: $\left(\left(\left(\frac{1}{7^0} + \frac{1}{7^1}\right) +
\frac{1}{7^2}\right) + \frac{1}{7^3}\right) + \frac{1}{7^4}$.
%
Temos:
{\tiny
\begin{align*}
  & 
  \left(
    \left(
      \left(\machinenumber{1.0000}{0} + \machinenumber{1.4286}{-1}\right) +
      \machinenumber{2.0408}{-2}
    \right) +
    \machinenumber{2.9155}{-3}
  \right) +
  \machinenumber{4.1649}{-4}\\
= &
  \left(
    \left(
      \machinenumber{1.1429}{0} + \machinenumber{2.0408}{-2}
    \right) +
    \machinenumber{2.9155}{-3}
  \right) +
  \machinenumber{4.1649}{-4}\\
= &
  \left(
      \machinenumber{1.1633}{0} + \machinenumber{2.9155}{-3}
  \right) +
  \machinenumber{4.1649}{-4}\\
= & \machinenumber{1.1662}{0} + \machinenumber{4.1649}{-4}\\
= & \alert<2->{\machinenumber{1.1666}{0}}
\end{align*}
}

Em (ii), calculamos: $\frac{1}{7^0} + \left(\frac{1}{7^1} +
\left(\frac{1}{7^2} + \left(\frac{1}{7^3} + \frac{1}{7^4}\right)\right)\right)$.
%
Temos:
{\tiny
\begin{align*}
  & 
  \machinenumber{1.0000}{0} + \left(\machinenumber{1.4286}{-1} +
  \left(\machinenumber{2.0408}{-2} + \left(\machinenumber{2.9155}{-3} +
  \machinenumber{4.1649}{-4}\right)\right)\right)\\
%
  =& 
  \machinenumber{1.0000}{0} + 
  \left(\machinenumber{1.4286}{-1} +
    \left(\machinenumber{2.0408}{-2} + 
      \machinenumber{3.3319}{-3}
    \right)
  \right)\\
  =& 
  \machinenumber{1.0000}{0} + 
  \left(\machinenumber{1.4286}{-1} +
    \machinenumber{2.3740}{-2}  
  \right)\\
  =& \machinenumber{1.0000}{0} + \machinenumber{1.6660}{-1} \\
  =& \alert<2->{\machinenumber{1.1666}{0}}
\end{align*}
}

\end{frame}

\subsubsection{Questão 1.11.e}

\begin{frame}
\frametitle{Questão 1.11.e}
\framesubtitle{Representação numérica, aritmética de ponto flutuante}

Considere o sistema de ponto flutuante dado por $\machine{10}{6}{-99}{99}$. Os
elementos $x = \machinenumber{0.4721025}{8}$, $y = \machinenumber{1.00321}{5}$ e
$z = \machinenumber{0.0072134}{6}$ pertencem a essa máquina. Verifique usando as
representações de $x$, $y$ e $z$ neste sistema, se $x \cdot (y + z) = x \cdot y
+ x \cdot z$.

\end{frame}

\begin{frame}
\frametitle{Questão 1.11.e}
\framesubtitle{$\machine{10}{6}{-99}{99}$, $x = \machinenumber{0.4721025}{8} $,
$y = \machinenumber{1.00321}{5} $, $z = \machinenumber{0.0072134}{6} $}

Normalizando os números, temos:
\begin{align*}
x = \machinenumber{0.4721025}{8} =& \alert<2>{\machinenumber{4.72102}{7}}
\onslide<2->{\text{ [arredondamento]}}\\
y = \machinenumber{1.00321}{5} =& \machinenumber{1.00321}{5}\\
z = \machinenumber{0.0072134}{6} =& \machinenumber{7.21340}{3}\\
\end{align*}

\end{frame}

\begin{frame}
\frametitle{Questão 1.11.e}
\framesubtitle{$\machine{10}{6}{-99}{99}$, $x = \machinenumber{4.72102}{7} $,
$y = \machinenumber{1.00321}{5} $, $z = \machinenumber{7.21340}{3} $}

Fazendo $x \cdot (y + z)$, temos:
\begin{align*}
& \machinenumber{4.72102}{7} \cdot \left(\machinenumber{1.00321}{5} +
\machinenumber{7.21340}{3} \right) \\
= & \machinenumber{4.72102}{7} \cdot \machinenumber{1.07534}{5} \\
= & \alert<2->{\machinenumber{5.07670}{12}}
\end{align*}

Fazendo $x \cdot y + x \cdot z$, temos:
\begin{align*}
& \left(\machinenumber{4.72102}{7} \cdot \machinenumber{1.00321}{5}\right) 
  +
  \left(\machinenumber{4.72102}{7} \cdot \machinenumber{7.21340}{3} \right) \\
= &  \machinenumber{4.73617}{12} + \machinenumber{3.40546}{11}\\
= & \machinenumber{4.73617}{12} + \machinenumber{0.340546}{12}\\
= & \alert<2->{\machinenumber{5.07672}{12}}
\end{align*}

\onslide<2->{Logo, os cálculos têm resultados diferentes.}

\end{frame}

\section{Zeros de funções}
\subsection[Métodos de quebra e métodos iterativos]{Bisseção, falsa posição
(cordas), M.I.L., Newton e secantes}

\subsubsection{Questão 2.1}

\begin{frame}
\frametitle{Questão 2.1}
\framesubtitle{Bisseção, falsa posição (cordas), M.I.L., Newton e secantes}
{\small
Para cada função:
\begin{enumerate}
  \item Localizar, se existir, raiz real mais próxima da origem;
  \item Determinar analiticamente um intervalo de amplitude $0.1$ contendo tal
  raiz;
  \item Aplicar os métodos abaixo para calcular a raiz aproximada: 
  \begin{enumerate}{\footnotesize
    \item Bisseção\footnote{Para o método da Bisseção faça até que o intervalo
    de separação seja menor que $10^{-2}$ e indique quantas iterações serão
    necessárias antes de aplicar o método. Para os demais métodos, faça até que
    $|x_{i+1} - x_i| \leq 10^{-3}$ ou $i = 3$}
    \item Falsa posição (cordas)
    \item Iterativo linear
    \item Newton-Raphson
    \item Das secantes}
  \end{enumerate}
\end{enumerate}
Considere uma máquina com $5$ casas decimais e arredondamento padrão.
}

\end{frame}

\begin{frame}
\frametitle{Questão 2.1.g - localizar raiz real próxima à origem}
\framesubtitle{$\f\left(\x\right) = \x \cdot e^\x + \x^2 - 1$}

Vamos verificar alguns valores próximos da origem que facilitem os cálculos para
verificar a mudança de sinal:
\begin{center}
\begin{tabular}{r|r}
$\x$ & $\f\left(\x\right)$ \\
\hline
\hline
$\onslide<2->{-1}$ & $\onslide<3->{-\frac{1}{e}}$\\
\hline
$\onslide<4->{0}$ & $\onslide<5->{-1.00000}$\\
\hline
$\onslide<6->{1}$ & $\onslide<7->{e}$ \\
\hline
\end{tabular}
\end{center}

\onslide<7->{Há mudança de sinal no intervalo $[0,1]$, logo existe ao menos uma
raiz real.}
\end{frame}

\begin{frame}
\frametitle{Questão 2.1.g - encontrar o intervalo de separação de amplitude
$0.1$}
\framesubtitle{$\f\left(\x\right) = \x \cdot e^\x + \x^2 - 1$}

Como $\f$ é uma função crescente\footnote{$\f'\left(\x\right) =
\left(\x + 1\right)e^\x + 2 \cdot \x$} no intervalo $[0,1]$, calculamos os
valores de $\f$ com incremento de $0.1$ a partir de 0:
{\scriptsize
\begin{center}
\begin{tabular}{r|r}
$\x$ & $\f\left(\x\right)$ \\
\hline
\hline
$0$ & $-1.00000$\\
\hline
\onslide<2->{$0.1$} & \onslide<3->{$-8.79482908 \cdot 10^{-1}$}\\
\hline
\onslide<4->{$0.2$} & \onslide<5->{$-7.15719448 \cdot 10^{-1}$}\\
\hline
\onslide<6->{$0.3$} & \onslide<7->{$-5.05042358 \cdot 10^{-1}$}\\
\hline
\onslide<8->{$0.4$} & \onslide<9->{$-2.43270121 \cdot 10^{-1}$}\\
\hline
\onslide<10->{$0.5$} & \onslide<11->{$7.4360635 \cdot 10^{-2}$}\\
\hline
\end{tabular}
\end{center}
}

\onslide<11->{Logo, o intervalo de separação de amplitude $0.1$ é $[0.4,0.5]$.}
\end{frame}

\begin{frame}
\frametitle{Questão 2.1.g - calcular a quantidade de iterações no método da
bisseção}
\framesubtitle{$\f\left(\x\right) = \x \cdot e^\x + \x^2 - 1$, 
$\f'\left(\x\right) = \left(\x + 1\right)e^\x + 2 \cdot \x$, $\I =
[0.4,0.5]$, $l = 10^{-2}$}

A quantidade de iterações é dada pela fórmula:
\[
\k \geq \frac{\ln\left(b_0 - a_0\right) - \ln l}{\ln 2}
\]

Substituindo pelos valores:
\begin{align*}
\k &\geq \frac{\ln\left(\onslide<2->{0.5} - \onslide<3->{0.4}\right) - \ln
\onslide<4->{10^{-2}}}{\ln 2}\\
\k &\geq \onslide<5->{3.32193}
\end{align*}

\onslide<5->{A quantidade de iterações necessárias é $4$.}

\end{frame}

\begin{frame}
\frametitle{Questão 1.g - aplicar o método da bisseção }
\framesubtitle{$\f\left(\x\right) = \x \cdot e^\x + \x^2 - 1$,
$\f'\left(\x\right) = \left(\x + 1\right)e^\x + 2 \cdot \x$ 
$\I = [0.4,0.5]$, $l = 10^{-2}$, $t = 4$, $i_{max} = 3$}

\[
\x = \frac{a + b}{2}
\]

{\scriptsize
\begin{center}
\begin{tabular}{r|r|r|r|r|r|r}
 & $a$ & $b$ & $\f\left(a\right)$ & $\f\left(b\right)$ & $\x$ &
$\f\left(\x\right)$ \\
\hline
\hline
$0$ & 
  $0.4$ & 
  \alert<3>{$0.5$} &
  \onslide<2->{$\machinenumber{-2.43270}{-1}$} &
  \onslide<2->{\alert<3>{$\machinenumber{7.43606}{-2}$}} & 
  \alert<3>{$0.45$} &
  \onslide<2->{\alert<3>{$\machinenumber{-9.17595}{-2}$}}
\\
\hline
\onslide<4->{$1$} & 
  \onslide<4->{$0.45$} &
  \onslide<4->{\alert<6>{$0.5$}} & 
  \onslide<5->{$\machinenumber{-9.17595}{-2}$} &
  \onslide<5->{\alert<6>{$\machinenumber{7.43606}{-2}$}} & 
  \onslide<4->{\alert<6>{$0.475$}} &
  \onslide<5->{\alert<6>{$\machinenumber{-1.05683}{-2}$}}
\\
\hline
\onslide<6->{$2$} & 
  \onslide<6->{\alert<8>{$0.475$}} &
  \onslide<6->{$0.5$} &  
  \onslide<7->{\alert<8>{$\machinenumber{-1.05683}{-2}$}} &
  \onslide<7->{$\machinenumber{7.43606}{-2}$} &
  \onslide<6->{\alert<8>{$0.4875$}} &
  \onslide<7->{\alert<8>{$\machinenumber{3.14235}{-2}$}}
\\
\hline
\onslide<8->{$3$} & 
  \onslide<8->{\alert<10>{$0.475$}} &
  \onslide<8->{$0.4875$} &  
  \onslide<9->{\alert<10>{$\machinenumber{-1.05683}{-2}$}} &
  \onslide<9->{$\machinenumber{3.14235}{-2}$} &
  \onslide<8->{\alert<10>{$0.48125$}} &
  \onslide<9->{\alert<10>{$\machinenumber{1.03101}{-2}$}}
\\
\hline
\onslide<10->{$4$} & 
  \onslide<10->{$0.475$} &
  \onslide<10->{$0.48125$} & 
  \onslide<11->{$\machinenumber{-1.05683}{-2}$} &
  \onslide<11->{$\machinenumber{1.03101}{-2}$} &
  \onslide<10->{$0.478125$} & 
  \onslide<11->{$\machinenumber{-1.58339}{-4}$}
\\
\hline
\end{tabular}
\end{center}
}

\onslide<12->{$b_4 - a_4 = 0.00625 < 10^{-2}$ e $\x = 0.478125$}
\end{frame}

\begin{frame}
\frametitle{Questão 2.1.g - aplicar o método das cordas }
\framesubtitle{$\f\left(\x\right) = \x \cdot e^\x + \x^2 - 1$,
$\f'\left(\x\right) = \left(\x + 1\right)e^\x + 2 \cdot \x$ 
$\I = [0.4,0.5]$, $|\x_{i+1} - \x_i| \leq 10^{-3}$, $i_{max} = 3$}

\[
\x = \frac{a \cdot \f\left(b\right) - b \cdot
\f\left(a\right)}{\f\left(b\right) - \f\left(a\right)}
\]

{\tiny
\begin{center}
\begin{tabular}{r|r|r|r|r|r|r|r}
  & 
  $a$ & 
  $b$ &
  $\f\left(a\right)$ &
  $\f\left(b\right)$ &
  $\x_i$ &
  $\f\left(\x_i\right)$ &
  $|\x_{i+1} - \x_i|$
\\
\hline
\hline
  1 &
  $\machinenumber{4.00000}{-1}$ &
  $\machinenumber{5.00000}{-1}$ &
  $\machinenumber{-2.43270}{-1}$&
  $\machinenumber{7.43610}{-2}$&
  $\machinenumber{4.76589}{-1}$ &
  $\machinenumber{-5.28300}{-3}$&
  ---\\
\hline
  2 &
  $\machinenumber{4.76589}{-1}$&
  $\machinenumber{5.00000}{-1}$ &
  $\machinenumber{-5.28300}{-3}$ &
  $\machinenumber{7.43610}{-2}$ &
  $\machinenumber{4.78142}{-1}$ &
  $\machinenumber{-1.02000}{-4}$ &
  $\machinenumber{1.55292}{-3}$\\
\hline
  3 &
  $\machinenumber{4.78142}{-1}$&
  $\machinenumber{5.00000}{-1}$ &
  $\machinenumber{-1.02000}{-4}$ &
  $\machinenumber{7.43610}{-2}$ &
  $\machinenumber{4.78172}{-1}$ &
  $\machinenumber{-2.00000}{-6}$ &
  $\machinenumber{2.99415}{-5}$\\
\hline
\end{tabular}
\end{center}
}

Encontramos $\x = 0.478172$.

\end{frame}

\begin{frame}
\frametitle{Questão 2.1.g - aplicar o M.I.L. }
\framesubtitle{$\f\left(\x\right) = \x \cdot e^\x + \x^2 - 1$,
$\f'\left(\x\right) = \left(\x + 1\right)e^\x + 2 \cdot \x$ 
$\I = [0.4,0.5]$, $|\x_{i+1} - \x_i| \leq 10^{-3}$, $i_{max} = 3$}

{\footnotesize
Inicialmente precisamos determinar uma função geradora $\varphi$. Como $\x
\cdot e^\x + \x^2 - 1 = 0$, podemos escolher 
$\varphi_1\left(\x\right) = x = \frac{1 - \x^2}{e^\x}$. Sua derivada é: 
$\varphi_1'\left(\x\right) = 
    \frac{\left(x^2-1\right)\cdot e^\x - 2\x \cdot e^\x}
  {e^{2x}}$

Vamos verificar as condições de convergência:
\begin{enumerate}
  \item As duas funções $\varphi_1$ e $\varphi_1'$ são contínuas.
  \item $|\varphi_1'\left(\x\right)| \leq k < 1, \forall x \in \I$:\\
\begin{tabular}{r|r}
$\x$ & $|\varphi_1'\left(\x\right) |$ \\
\hline
\hline
$\machinenumber{4.0000}{-1}$ & $\machinenumber{1.09932}{0}$ \\
\hline
$\machinenumber{4.5000}{-1}$ & $\machinenumber{1.08237}{0}$ \\
\hline
$\machinenumber{5.0000}{-1}$ & $\machinenumber{1.06143}{0}$ \\
\hline
\end{tabular}
\end{enumerate}

Logo $\varphi_1'$ não pode ser usada. 

Trivialmente não podemos escolher. Tente obter $\varphi_2$ e $\varphi_3$
isolando os outros termos da equação.

}
\end{frame}

\begin{frame}
\frametitle{Questão 2.1.g - aplicar o método new Newton-Raphson}
\framesubtitle{$\f\left(\x\right) = \x \cdot e^\x + \x^2 - 1$,
$\f'\left(\x\right) = \left(\x + 1\right)e^\x + 2 \cdot \x$ 
$\I = [0.4,0.5]$, $|\x_{i+1} - \x_i| \leq 10^{-3}$, $i_{max} = 3$}

A função geradora é definida por: 
\[
\varphi\left(\x_i\right) = \x_{i+1} = \x_i -
\frac{\f\left(\x_i\right)}{\f'\left(\x_i\right)}
\]

Logo, temos que:
 
\[
\varphi\left(\x_i\right) = \x_{i+1} = \x_i -
\frac{\x_i \cdot e^\x_i + \x_i^2 - 1}
{\left(\x_i + 1\right)e^\x_i + 2 \cdot \x_i}
\]

Aplicando o método obtemos:\\
{\scriptsize
\begin{center}
\begin{tabular}{r|r|r|r}
  &
  $\x$ &
  $\varphi\left(\x\right)$ &
  $|\x_{i+1} - \x_i|$\\
\hline
\hline
0 & $\machinenumber{4.50000}{-1}$ & $\machinenumber{4.78909}{-1}$ & ---\\
\hline
1 & $\machinenumber{4.78909}{-1}$ & $\machinenumber{4.78173}{-1}$ 
  & $\machinenumber{7.36000}{-4}$\\
\hline
\end{tabular}
\end{center}
}

Encontramos $\x = 0.478173$.

\end{frame}

\begin{frame}
\frametitle{Questão 2.1.g - aplicar o método das secantes}
\framesubtitle{$\f\left(\x\right) = \x \cdot e^\x + \x^2 - 1$,
$\f'\left(\x\right) = \left(\x + 1\right)e^\x + 2 \cdot \x$ 
$\I = [0.4,0.5]$, $|\x_{i+1} - \x_i| \leq 10^{-3}$, $i_{max} = 3$}

A função geradora é definida por: 
\[
\varphi\left(\x_i\right) = \x_{i+1} = 
\frac
  {\x_{i-1} \cdot \f\left(\x_i\right) - \x_i \cdot \f \left(\x_{i-1}\right)}
  {\f\left(\x_i\right) - \f\left(\x_{i-1}\right)}
\]

Aplicando o método obtemos:\\
{\tiny
\begin{center}
\begin{tabular}{r|r|r|r|r|r|r}
  &
  $\x_i$ &
  $\x_{i-1} $ &
  $\f\left(\x_i\right) $ &
  $\f\left(\x_{i-1}\right) $ &
  $\varphi\left(\x_i\right) $ &
  $|\x_{i+1} - \x_i|$ \\
\hline
\hline
  0 &
  $\machinenumber{4.00000}{-1}$ &
  $\machinenumber{5.00000}{-1}$ &
  $\machinenumber{-2.43270}{-1}$ &
  $\machinenumber{7.43610}{-2}$ &
  $\machinenumber{4.76589}{-1}$ &
  ---
  \\
\hline
  1 &
  $\machinenumber{4.76589}{-1}$ &
  $\machinenumber{4.00000}{-1}$ &
  $\machinenumber{-5.28000}{-3}$ &
  $\machinenumber{-2.43270}{-1}$ &
  $\machinenumber{4.78289}{-1}$ &
  $\machinenumber{1.70000}{-3}$
  \\
\hline
  2 &
  $\machinenumber{4.78289}{-1}$ &
  $\machinenumber{4.76589}{-1}$ &
  $\machinenumber{3.90000}{-4}$ &
  $\machinenumber{-5.28200}{-3}$ &
  $\machinenumber{4.78172}{-1}$ &
  $\machinenumber{1.17000}{-4}$
  \\
\hline
\end{tabular}
\end{center}
}

Encontramos $\x = 0.478172$.
\end{frame}

