\section{Introdução}

\begin{frame}
\frametitle{Introdução}

\begin{itemize}
  \item Quando estudamos um fenômeno de forma experimental, é comum termos um conjunto de valores tabelados
  \item Utilizando tais informações podemos levantar várias questões
    \begin{itemize}
      \item Qual a relação existente entre $\x$ e $\f\left(x\right)$?
      \item Qual o valor de $\f\left(x\right)$ para um determinado $\x$ fora do tabelamento?
    \end{itemize}
\end{itemize}
\end{frame}

\begin{frame}
\frametitle{Introdução}

\begin{itemize}
  \item Nestas circustâncias, temos um tabelamento da forma:
%
\begin{center}
\onslide<2->{
\begin{tabular}{|c|c|c|c|c|}
\hline
$\x_i$ & $\x_0$ & $\x_1$ & \ldots & $\x_n$\\
\hline
$\f\left(\x_i\right)$ & $\f\left(\x_0\right)$ & $\f\left(\x_1\right)$ & \ldots & $\f\left(\x_n\right)$\\
\hline
\end{tabular}
}
\end{center}
%
  \item Como podemos usar o tabelamento para calcular o valor da função $\f$ desconhecida em pontos não tabelados?
  \item $\f\left(\x\right)$ mapeia algum fenômeno com dados colhidos de forma experimental
    \begin{itemize}
      \item<3-> Não temos certeza sobre corretude dos dados colhidos
    \end{itemize}
\end{itemize}

\end{frame}

\begin{frame}
\frametitle{Aplicações}

\begin{itemize}[<+->]
  \item Previsão para o estoque de um determinado produto em função do histórico de sua demanda

  \item Previsão de inflação
  \item Consumo energético
  \item Dados polucionais etc.
\end{itemize}
\end{frame}

\begin{frame}
\frametitle{Ajustamento de curvas}

\begin{itemize}
  \item Definição: o problema de ajuste de curvas no caso em que temos um tabelamento de pontos 
$\left(\x_0, \f\left(\x_0\right)\right), \left(\x_1, \f\left(\x_1\right) \right), \ldots, \left(\x_n, \f\left(\x_n\right) \right) $, com $\x_0, \x_1, \ldots, \x_n \in [a,b]$, consiste em:
  
  \item Escolhidas $m+1$ funções 
$\G_0\left(\x\right), \G_1\left(\x\right), \ldots, \G_m\left(\x\right) $, contínuas em $[a,b]$, obter $m+1$ constantes $a_0, a_1, \ldots, a_m$, tais que a função $P\left(\x \right) = a_0 \G_0\left(\x\right) + a_1 \G_1 \left(\x\right) + \ldots + a_m \G_m \left(\x\right) $ se aproxime ao máximo de $\f\left(\x\right) $
\end{itemize}
\end{frame}

\begin{frame}
\frametitle{Ajustamento de curvas}

\begin{itemize}
  \item Temos uma combinação linear de funções elementares
  \item $P\left(\x\right) = \sum_{j=0}^m a_j \G_j \left(x\right) $
    \begin{itemize}
      \item $a_j$: coeficientes a serem ajustados
      \item $\G_j$: funções conhecidas ($1$, $\x$, $\sen x$, $\ln x$)
    \end{itemize}
  \item Desejamos escolher a função $P\left(\x\right)$ que melhor represente o tabelamento utilizado
\end{itemize}
\end{frame}

\begin{frame}
\frametitle{Ajustamento de curvas}

\begin{itemize}
  \item Como escolher as funções contínuas $\G_0\left(\x\right), \G_1\left(\x\right), \ldots, \G_m\left(\x\right) $?
  \item Uma maneira simples consiste em analisar os pontos conhecidos em um gráfico cartesiano
  \item Ex.:
    \begin{multicols}{2}
      \begin{itemize}
        \item $\G_1\left(\x\right) = x^2$
        \item Procuramos o valor de a em: $\varphi \left(\x\right) = a\x^2$
        \item Ou seja, qual a parábola com equação $a\x^2$ melhor se ajusta aos dados?
        \item<2-> Esta escolha nem sempre é simples e não será objeto de estudo neste curso. 
      \end{itemize}    
      \columnbreak
      \begin{tikzpicture}[global scale=0.7]
      
      \draw plot[mark=*,only marks, xshift=3cm, yshift=3cm] file {ajustamento-dados1.txt};
      \onslide<2->{%
      \draw[domain=-1:1, smooth, variable=\x, xshift=3cm, yshift=3cm] plot ({\x},{2*\x*\x});
      \node[right] at (4,5) {$\f\left(\x\right) = 2\x^2$};
      }
      
      \draw[->] (0,3) -- coordinate (x axis mid) (6,3) node[right] {$\x$};
      \draw[->] (3,0) -- coordinate (y axis mid) (3,6) node[above] {$\f\left(\x\right) $};
      \foreach \x in {-1,0,1}
        \draw [xshift=3cm,yshift=3cm](\x,1pt) -- (\x,-3pt)
            node[anchor=north] {$\x$};
      \foreach \y/\ytext in {-2,-1,,1,2}
          \draw[xshift=3cm,yshift=3cm] (1pt,\y) -- (-3pt,\y) node[anchor=east] {$\ytext$};
          
      \end{tikzpicture}
    \end{multicols}
\end{itemize}

\end{frame}

\begin{frame}
\frametitle{Ajustamento de curvas}

\begin{itemize}
  \item O que significa obter uma curva que melhor se ajuste, ou que mais se aproxime de uma função $\f\left(\x\right)$ desconhecida?
  \item Ideia geométrica:\\
    \begin{tikzpicture}[global scale=0.7]
    
    %\draw plot[mark=*,only marks, xshift=0cm, yshift=0cm] file {ajustamento-dados2.txt};
    %\draw plot[xshift=0cm, yshift=0cm] file {ajustamento-dados3.txt};
    %\draw[->] (0,0) -- coordinate (x axis mid) (6,0) node[right] {$\x$};
    %\draw[->] (0,0) -- coordinate (y axis mid) (0,6) node[above] {$y $};
    \begin{axis}[%
      xmin=0, xmax=6, ymin=0, ymax=6, axis x line=center, axis y line=center
    ]
      \addplot plot[mark=*,only marks, xshift=0cm, yshift=0cm] file {ajustamento-dados2.txt};
      \addplot plot[mark=none, xshift=0cm, yshift=0cm] file {ajustamento-dados3.txt};
      
    \end{axis}

    \onslide<2->{%
    \draw [decorate, decoration={brace, amplitude=2pt, mirror}, xshift=13pt, yshift=-5pt] (3,3.2) -- (3,3.7) node [midway,xshift=55pt] {%
    $R\left(\x_i\right) = P\left(\x_i\right) - \f\left(\x_i\right)$
    };
    }
    
    \onslide<2->{\draw (5,5) node {Objetivo: tornar os resíduos $R\left(\x_i\right) mínimos$};}
    \end{tikzpicture}
\end{itemize}
\end{frame}

\begin{frame}
\frametitle{Ajustamento de curvas}

\begin{itemize}
  \item O que significa tornar os resíduos $R\left(\x_i\right)$ mínimos?
  \item $\sum_{i=0}^n R\left(\x_i\right) = 0$ ?
    \begin{itemize}
      \item Não! A curva pode ter resíduos positivos e negativos grandes em valores absolutos, mas que somados se aproximem bastante de zero. Escolha inadequada.
    \end{itemize}
  \item $\sum_{i=0}^n \left|R\left(\x_i\right)\right| = 0$ ?
    \begin{itemize}
      \item Não! Função valor absoluto não é derivável em seu mínimo. 
    \end{itemize}  
  \item $\sum_{i=0}^n R^2\left(\x_i\right) = 0$ ?
    \begin{itemize}
      \item Sim! Problemas anteriores são resolvidos
      \item Buscaremos a função do tipo escolhido que produza a menor soma dos quadrados dos resíduos
      \item Método dos mínimos quadrados (MMQ)
    \end{itemize}
\end{itemize}
\end{frame}

\begin{frame}
\frametitle{Método dos mínimos quadrados}

\begin{itemize}
  \item Função $\varphi$ associa a função escolhida para representar a tabela dada à soma dos quadrados dos resíduos produzidos por ela
  \item Procuramos o mínimo de\\
  %
  \[\varphi\left(a_0,a_1,\ldots,a_m\right) = \sum_{i=0}^{n}R^2\left(\x_i\right)\]
\end{itemize}
\end{frame}

\begin{frame}
\frametitle{Método dos Mínimos Quadrados}

\begin{itemize}
  \item Para o caso específico de uma reta teremos:
    \begin{itemize}
      \item $P\left(\x\right) = a_0 \G_0 \left(\x\right) + a_1 \G_1 \left(\x\right)$,
      \item onde $\G_0\left(\x\right) = 1$ e $\G_1\left(x\right) = x$
    \end{itemize}
  \item Teremos para cada possível par $\left(a_0, a_1\right)$ uma reta $P_0 \left(\x\right) = a_0 + a_1 \x$ distinta
  \item Em uma delas a diferença para a função $\f$ será mínima.
\end{itemize}
\end{frame}

\begin{frame}
\frametitle{Método dos Mínimos Quadrados}

\begin{itemize}
  \item Queremos, portanto, encontrar os valores $\left(a_0, a_1, \ldots, a_m\right)$ que minimizem a soma do quadrado dos resíduos: $\min \varphi\left(a_0,a_1,\ldots,a_m\right) = \sum_{i=0}^{n}R^2\left(\x_i\right)$
  \item Então, temos que:
\begin{align}
\frac{\partial \varphi}{\partial a_j} &= 0,\, j = 0,1,2,\ldots,m\label{eq:mmq_partial_1}\\
%
\frac{\partial \varphi}{\partial a_j} = %
\frac{\partial}{\partial a_j} \sum_{i=0}^{n} R^2 \left(\x_i\right) = %
\sum_{i=0}^{n} \frac{\partial R^2 \left(\x_i\right)}{\partial a_j} &= %
2 \sum_{i=0}^{n} R\left(\x_i\right) \frac{\partial R \left(\x_i\right)}{\partial a_j}\label{eq:mmq_partial_2}
\end{align}
%
\end{itemize}
\end{frame}

\begin{frame}
\frametitle{Método dos Mínimos Quadrados}

Como
{
\footnotesize
\begin{equation}
\label{eq:mmq_partial_3}
R\left(x_i\right) = P\left(x_i\right) - \f\left(x_i\right) =% 
\alert<2>{a_0 \G_0 \left(\x_i\right) + \ldots + a_j \G_j \left(x_i\right) + \ldots + a_m \G_m \left(x_i\right) - \f \left(\x_i\right)}
\end{equation}}%
temos que:\\
$\frac{\partial R \left(\x_i\right)}{\partial a_j} = \G_j \left(\x_i\right)$, logo de \cref{eq:mmq_partial_2} temos que:
\[
\frac{\partial \varphi}{\partial a_j} = 2 \sum_{i=0}^{n} R\left(\x_i\right) \G_j \left(\x_i\right), j = 0,1,\ldots,m
\]
Portanto, considerando que $\frac{\partial \varphi}{\partial a_j} = 0$ (\cref{eq:mmq_partial_1}), temos o sistema normal dado por:
\[
\sum_{i=0}^{n} \alert<2>{R\left(\x_i\right)} \G_j \left(\x_i\right) = 0, j = 0, 1, \ldots, m
\]
\end{frame}

\begin{frame}
\frametitle{Método dos Mínimos Quadrados}

Substituindo $R\left(\x_i\right)$ conforme \cref{eq:mmq_partial_3}, o sistema normal pode ser reescrito como
\begin{align*}
&a_0 \sum_{i=0}^n \G_0\left(\x_i\right) \G_j\left(x_i\right) +
a_1 \sum_{i=0}^n \G_1\left(\x_i\right) \G_j\left(x_i\right) + \ldots +\\
%
&a_m \sum_{i=0}^n \G_m\left(\x_i\right) \G_j\left(x_i\right) =
\sum_{i=0}^n \f\left(\x_i\right) \G_j\left(x_i\right), j = 0,1,2,\ldots,m
\end{align*}%

De forma mais sintética:
\[
\sum_{k=0}^m a_k \sum_{i=0}^n \G_k\left(\x_i\right) \G_j\left(\x_i\right) = 
\sum_ {i=0}^n \f \left(x_i\right) \G_j\left(\x_i\right), j = 0, 1, 2, \ldots, m
\]

\onslide<2->{Este sistema possui solução única e essa solução é o ponto de mínimo de $\varphi \left(a_0, a_1, \ldots, a_m\right)$}
\end{frame}

\begin{frame}
\frametitle{MMQ -- Exemplo}

Considere as taxas de inflação no período de janeiro a setembro de um certo ano dadas pela tabela abaixo. Faça uma previsão para os meses de outubro a dezembro desse mesmo ano considerando que uma reta é o tipo de curva que melhor representa esse fenômeno.

\begin{center}
\begin{tabular}{|c|c|c|c|c|c|c|c|c|c|}
\hline
Mês & Jan & Fev & Mar & Abr & Mai & Jun & Jul & Ago & Set\\
\hline
Inflação & 1,3 & 1,8 & 2,2 & 0,4 & 1,1 & 3,0 & 1,1 & 0,8 & 0,1\\
\hline
\end{tabular}
\end{center}

\onslide<2->{
O que se quer é encontrar a reta que melhor se ajusta à tabela. Logo:
\[
P\left(\x\right) = \sum_{j=0}^1 a_j \G_j\left(x\right) = 
a_0 \G_0 \left(\x\right) + a_1 \G_1 \left(x\right)
\]
onde $\G_0\left(x\right) = 1$ e $\G_1\left(\x\right) = \x$,
ou ainda,
\[
P \left(\x\right) = a_0 + a_1 x
\]
}
\end{frame}

\begin{frame}
\frametitle{MMQ -- Exemplo}

Para efeito de cálculos, temos: $jan = 1$, $fev = 2$ e assim por diante. Como $m=1$

Da equação
\[
\sum_{k=0}^m a_k \sum_{i=0}^n \G_k\left(\x_i\right) \G_j\left(\x_i\right) = 
\sum_ {i=0}^n \f \left(x_i\right) \G_j\left(\x_i\right), j = 0, 1, 2, \ldots, m
\]

Obtemos o sistema
\begin{align*}
\onslide<3->{
a_0 \sum_{i=0}^n \G_0 \left(x_i\right) \G_0 \left(x_i\right) + 
a_1 \sum_{i=0}^n \G_1 \left(x_i\right) \G_0 \left(x_i\right) &=
\sum_{i=0}^n \f \left(x_i\right) \G_0 \left(x_i\right)} & \onslide<2->{j = 0}\\
%
\onslide<5->{a_0 \sum_{i=0}^n \G_0 \left(x_i\right) \G_1 \left(x_i\right) + 
a_1 \sum_{i=0}^n \G_1 \left(x_i\right) \G_1 \left(x_i\right) &=
\sum_{i=0}^n \f \left(x_i\right) \G_1 \left(x_i\right)} & \onslide<4->{j = 1}
\end{align*}

\end{frame}

\begin{frame}
\frametitle{MMQ -- Exemplo}

Calculando os coeficientes
{
\tiny
\begin{center}
\begin{tabular}{c|c|c|c|c|c|c|c|c|c}
\hline
$i$ & $\x_i$ & $\f\left(\x_i\right)$ & 
  $\G_0\left(\x_i\right)$ &
  $\G_1\left(\x_i\right)$ &
  $\G_0^2\left(\x_i\right)$ &
  $\G_1^2\left(\x_i\right)$ &
  $\G_0\left(\x_i\right) \G_1\left(\x_i\right)$ &
  $\f\left(\x_i\right) \G_0\left(\x_i\right)$ &
  $\f\left(\x_i\right) \G_1\left(\x_i\right)$\\
\hline
0 & 1 & 1,3 & 1 & 1 & 1 & 1 & 1 & 1,3 & 1,3\\
\hline
1 & 2 & 1,8 & 1 & 2 & 1 & 4 & 2 & 1,8 & 3,6\\
\hline
2 & 3 & 2,2 & 1 & 3 & 1 & 9 & 3 & 2,2 & 6,6\\
\hline
3 & 4 & 0,4 & 1 & 4 & 1 & 16 & 4 & 0,4 & 1,6\\
\hline
4 & 5 & 1,1 & 1 & 5 & 1 & 25 & 5 & 1,1 & 5,5\\
\hline
5 & 6 & 3,0 & 1 & 6 & 1 & 36 & 6 & 3,0 & 18,0\\
\hline
6 & 7 & 1,1 & 1 & 7 & 1 & 49 & 7 & 1,1 & 7,7\\
\hline
7 & 8 & 0,8 & 1 & 8 & 1 & 64 & 8 & 0,8 & 6,4\\
\hline
8 & 9 & 0,1 & 1 & 9 & 1 & 81 & 9 & 0,1 & 0,9\\
\hline 
& & & & $\sum$ & 9 & 285 & 45 & 11,8 & 51,6\\
\hline
\end{tabular}
\end{center}
}

\onslide<2->{
Chegamos ao sistema:
\[
\begin{cases}
9 a_0 + 45 a_1 &= 11,8\\
45 a_0 + 285 a_1 &= 51,6
\end{cases}
\]
Cuja solução é: $a_0 = 1,928$ e $a_1 = -0,123$, logo $P\left(\x\right) = 1,928 - 0,123x$
}
\end{frame}

\begin{frame}
\frametitle{MMQ -- Exemplo}
E respondendo à questão:

\begin{tabular}{c|c|c}
\hline
Mês & ``equivalência'' $\Rightarrow x$ & Valor aprox. da inflação $P\left(x\right)$\\
\hline
outubro & 10 & 0,698 \\
\hline
novembro & 11 & 0,575 \\
\hline
dezembro & 12 & 0,452 \\
\hline
\end{tabular}
\end{frame}

\begin{frame}
\frametitle{MMQ -- Exercício 4.1b}

Determine $P \left(\x\right) = a \x^2 + b$ que melhor se ajuste à tabela abaixo:

\begin{center}
\begin{tabular}{|c|c|c|c|c|}
\hline
$x_i$ & $1$ & $2$ & $3$ & $5$\\
\hline
$\f\left(x_i\right)$ & $1,2$ & $0,9$ & $3,1$ & $6,1$\\
\hline 
\end{tabular}
\end{center}

\onslide<2->{Reposta: $P\left(x\right) = 0,219 \x^2 + 0,691$}
\end{frame}

\section{Caso não linear de ajustamento}

\begin{frame}
\frametitle{Caso não linear}

\begin{itemize}
  \item Para aplicarmos o MMQ é necessário que $P$ seja linear nos parâmetros
  \item Quando isso não ocorre devemos fazer uma mudança de variável para \emph{tentar tornar} o problema em um problema de ajuste linear 
\end{itemize}

\end{frame}

\begin{frame}
\frametitle{Caso não linear}

\begin{itemize}
  \item Ex.: Encontre a curva do tipo $P\left(\x\right) = a_0 e^{a_1 x}$ que melhor se ajuste à tabela abaixo usando o MMQ
\begin{tabular}{|c|c|c|c|c|c|c|c|c|c|c}
\hline
Mês & Jan & Fev & Mar & Abr & Mai & Jun & Jul & Ago & Set\\
\hline
Inflação & $1,3$ & $1,8$ & $2,2$ & $0,4$ & $1,1$ & $3,0$ & $1,1$ & $0,8$ & $0,1$\\
\hline
\end{tabular}
  \item Trabalharemos com 
    \begin{itemize}
      \item $\ln \left( P\left(x\right) \right) = \ln \left(a_0 e^{a_1 x}\right) = \ln a_0 + a_1 x = a_0' + a_1 x$
      \item $a_0' = \ln a_0$, $\G_0\left(x\right) = 1$, $\G_1\left(x\right) = x$
    \end{itemize}
  \item Reconstruindo a tabela com $\ln \left( \f \left(x\right) \right)$
\begin{tabular}{|c|c|c|c|c|c|c|c|c|c|c}
\hline
Mês & Jan & Fev & Mar & Abr & Mai & Jun & Jul & Ago & Set\\
\hline
Inflação & $\ln 1,3$ & $\ln 1,8$ & $\ln 2,2$ & $\ln 0,4$ & $\ln 1,1$ & $\ln 3,0$ & $\ln 1,1$ & $\ln 0,8$ & $\ln 0,1$\\
\hline
\end{tabular}\end{itemize}
\end{frame}

\begin{frame}
\frametitle{Caso não linear}

\begin{itemize}
  \item Aplicando o sistema normal, com $m=1$, temos:
\begin{align*}
a_0' \sum_{i=0}^{n} \G_0\left(\x_i\right) \G_0\left(\x_i\right) + 
  a_1 \sum_{i=0}^{n} \G_1\left(\x_i\right) \G_0\left(\x_i\right) &=
  \sum_{i=0}^{n} \ln \left(\f\left(\x_i\right)\right) \G_0 \left(\x_i\right)\\
%
a_0' \sum_{i=0}^{n} \G_0\left(\x_i\right) \G_1\left(\x_i\right) + 
  a_1 \sum_{i=0}^{n} \G_1\left(\x_i\right) \G_1\left(\x_i\right) &=
  \sum_{i=0}^{n} \ln \left(\f\left(\x_i\right)\right) \G_1 \left(\x_i\right)
\end{align*}
  \item Onde $\G_0 \left(\x\right) = 1$, $\G_1 \left(\x\right) = \x$ e $a_0' = \ln a_0$
\end{itemize}
\end{frame}

\begin{frame}
\frametitle{Caso não linear}

Precisamos encontrar cada um dos coeficientes utilizados no sistema anterior. Para isso, construímos a tabela abaixo.

\footnotesize
\begin{tabular}{c|c|c|c|c|c|c|c|c|c}
$i$ & 
  $\x_i$ & 
  $\ln \left( \f \left(\x_i\right) \right)$ &
  $\G_0 \left(\x_i\right)$ &
  $\G_1 \left(\x_i\right)$ &
  $\G_0^2 \left(\x_i\right)$ &
  $\G_1^2 \left(\x_i\right)$ &
  $\G_0 \left(\x_i\right) \G_1 \left(\x_i\right)$ &
  $\ln \left(\f \left(\x_i\right) \right) \G_0 \left(\x_i\right)$ &
  $\ln \left(\f \left(\x_i\right) \right) \G_1 \left(\x_i\right)$\\
\hline
$0$ & $1$  & $0,262$   & $1 $ & $1 $ & $1 $ & $1 $  & $1 $ & $0,262$ & $0,262$\\
\hline
$1$ & $2 $ & $0,588 $  & $1 $ & $2 $ & $1 $ & $4 $  & $2 $ & $0,588 $ & $1,176 $\\
\hline
$2$ & $3 $ & $0,788 $  & $1 $ & $3 $ & $1 $ & $9 $  & $3 $ & $0,788 $ & $2,364 $\\
\hline
$3$ & $4 $ & $-0,916 $ & $1 $ & $4 $ & $1 $ & $16 $ & $4 $ & $-0,916 $ & $-3,664 $\\
\hline
$4$ & $5 $ & $0,095 $  & $1 $ & $5 $ & $1 $ & $25 $ & $5 $ & $0,095 $ & $0,475 $\\
\hline
$5$ & $6 $ & $1,099 $  & $1 $ & $6 $ & $1 $ & $36 $ & $6 $ & $1,099 $ & $6,594 $\\
\hline
$6$ & $7 $ & $0,095 $  & $1 $ & $7 $ & $1 $ & $49 $ & $7 $ & $0,095 $ & $0,665 $\\
\hline
$7$ & $8 $ & $-0,223 $ & $1 $ & $8 $ & $1 $ & $64 $ & $8 $ & $-0,223 $ & $-1,784 $\\
\hline
$8$ & $9 $ & $-2,303 $ & $1 $ & $9 $ & $1 $ & $81 $ & $9 $ & $-2,303 $ & $-20,727 $\\
\hline
\onslide<2->{& & & & $\sum$ & $9 $ & $285 $ & $45 $ & $-0,515 $ & $-14,639 $}
\end{tabular}
\end{frame}

\begin{frame}
\frametitle{Caso não linear}

\begin{itemize}
  \item Logo, chegamos ao sistema
\[
\begin{cases}
9 a_0' + 45 a_1 &= -0,515 \\
45 a_0' + 285 a_1 &= -14,639
\end{cases}
\]
  \item Cuja solução é $a_0' = 0,948$ e $a_1 = -0,201$, portanto $a_0' = \ln \left(a_0\right) \implies a_0 = e^{a_0'} = e^{0,948} = 2,581$
  \item $P \left(\x\right) = a_0 e^{a_1 \x} \implies P \left(\x\right) = 2,581 e^{-0,201 \x}$
\end{itemize}
\end{frame}

\begin{frame}
\frametitle{Exemplo 4.3 -- caso não linear}

Determine $P\left(x\right)$ como uma função do segundo grau que admita raízes $1$ e $2$, que melhor se ajuste à tabela a seguir usando MMQ com três casa decimais.

\begin{tabular}{c|c|c|c|c}
$x_i$ & $-1$ & $0$ & $2$ & $3$\\
\hline
$\f \left(\x_i\right)$ & $-4,0$ & $-1,5$ & $0,5$ & $0,0$ 
\end{tabular}

\onslide<2->{%
Vamos então fazer $\G_0 \left(x\right) = (\x-1)(\x-2)$, logo $P \left(\x\right) = a \left(\x-1\right) \left(\x-2\right) = a (\x^2 - 3\x + 2)$

Desta forma, o sistema normal reduz-se a:
\[
a_0 \sum_{i=0}^{n} \G_0 \left(\x_i\right) \G_0 \left(x_i\right) = \sum_{i=0}^{n} \f \left(\x_i\right) \G_0 \left(x_i\right)
\] 
}

\end{frame}

\begin{frame}
\frametitle{Exemplo 4.3 -- caso não linear}

Montamos então a tabela:

\begin{tabular}{c|c|c|c|c|c}
$i$ & $\x_i$ & $\f \left(\x_i\right)$ & $\G_0 \left(\x_i\right)$ & $\G_0^2 \left(\x_i\right)$ & $\f \left(\x_i\right) \G_0 \left(\x_i\right)$\\
\hline
$0$ & $-1 $ & $-4,0 $ & $6,0 $ & $36,0 $ & $-24,0 $\\
\hline
$1$ & $0 $ & $-1,5 $ & $2,0 $ & $4,0 $ & $-3,0 $\\
\hline
$2$ & $2 $ & $0,5 $ & $0,0 $ & $0,0 $ & $0,0 $\\
\hline
$3$ & $3 $ & $0,0 $ & $2,0 $ & $4,0 $ & $0,0 $\\
\hline
$$ & $ $ & $ $ & $ \sum $ & $44,0 $ & $-27,0 $\\
\end{tabular}

Então, temos a equação $44 a = -27 \implies a = -0,614$, logo $P \left(\x\right) = -0,614 \left(\x^2 - 3 \x + 2\right) \implies P \left(\x\right) = -0,614 \x^2 + 1,842 x - 1,228$
\end{frame}

\begin{frame}
\frametitle{Exercícios 4.1 \{a,g,i\} }
Usando o MMQ encontre a curva de cada uma das formas abaixo para a seguinte tabela:

\begin{tabular}{c|c|c|c|c}
$x_i$ & $1$ & $2$ & $3$ & $5$\\
\hline
$\f \left(\x_i\right)$ & $1,2$ & $0,9$ & $3,1$ & $6,1$\\
\hline
\end{tabular}

\begin{align*}
a) & P\left(\x\right) = a \x + b\\
%
g) & P\left(\x\right) = \frac{1}{a \x + b}\\
%
i) & P\left(\x\right) = a b^\x
\end{align*}

\end{frame}

\begin{frame}
\frametitle{Exercício 4.1g}
\framesubtitle{$P\left(\x\right) = \frac{1}{a\x + b}$}
\scriptsize

\begin{itemize}[<+->]
  \item Precisamos linearizar $P \left(\x \right)$.
  \item Aplicando a potência $-1$ a $P \left(\x \right)$, temos: $\left(P\left(\x\right)\right)^{-1} = \left(\frac{1}{a \x + b}\right)^{-1} = a \x + b$
  \item Vamos resolver o sistema com $\G_0\left(\x\right) = 1$ e $\G_1 \left(\x\right) = x$
  \item Montamos a tabela:
\begin{tabular}{c|c|c|c|c|c|c|c|c|c}
$i$ & 
  $\x_i$ & 
  $\left( \f \left(\x_i\right) \right)^{-1}$ &
  $\G_0 \left(\x_i\right)$ &
  $\G_1 \left(\x_i\right)$ &
  $\G_0^2 \left(\x_i\right)$ &
  $\G_1^2 \left(\x_i\right)$ &
  $\G_0 \left(\x_i\right) \G_1 \left(\x_i\right)$ &
  $\left(\f \left(\x_i\right) \right)^{-1} \G_0 \left(\x_i\right)$ &
  $\left(\f \left(\x_i\right) \right)^{-1} \G_1 \left(\x_i\right)$\\
\hline
$0$ & $1$ & $\frac{1}{1,2}$ & $1$ & $1$ & $1$ & $1$  & $1$ & $\frac{1}{1,2}$ & $\frac{1}{1,2}$\\
\hline
$1$ & $2$ & $\frac{1}{0,9}$ & $1$ & $2$ & $1$ & $4$  & $2$ & $\frac{1}{0,9}$ & $\frac{2}{0,9}$\\
\hline
$2$ & $3$ & $\frac{1}{3,1}$ & $1$ & $3$ & $1$ & $9$  & $3$ & $\frac{1}{3,1}$ & $\frac{3}{3,1}$\\
\hline
$3$ & $5$ & $\frac{1}{6,1}$ & $1$ & $5$ & $1$ & $25$ & $5$ & $\frac{1}{6,1}$ & $\frac{5}{6,1}$\\
\hline
& & & & $\sum$ & $4$ & $39$ & $11$ & $2,430$ & $4,841$  
\end{tabular}
  \item Obtemos o sistema:
\[
\begin{cases}
4 a + 11 b &= 2,430\\
%
11 a + 39 b &= 4,841
\end{cases}
\]
  \item Cuja solução é $a = 1,186$ e $b = -0,210$ e, portanto $P\left(\x\right) = \frac{1}{1,186 \x -0,210}$
\end{itemize}

\end{frame}

\begin{frame}
\frametitle{Exercício 4.1i}
\framesubtitle{$P\left(\x\right) = a b^x$}
\scriptsize
\begin{itemize}[<+->]
  \item Precisamos linearizar $P \left(\x \right)$.
  %\item Fazendo $a_1 = \ln b$, portanto $b = e^{a_1}$, $P\left(\x \right) = ab^x = a \left(e^{a_1}\right)^\x = a e^{a_1\x}$
  \item Aplicando o $\ln$ a $P \left(\x \right)$, temos: $\ln P\left(\x\right) = \ln \left(a b ^ x\right) = \ln a + \x \ln b$
  \item Fazendo $a_0 = \ln a$ e $a_1 = \ln b$, vamos resolver o sistema para $\ln P \left(\x\right) = a_0 + a_1 x$, com $\G_0\left(\x\right) = 1$ e $\G_1 \left(\x\right) = x$
  \item Montamos a tabela:
\begin{tabular}{c|c|c|c|c|c|c|c|c|c}
$i$ & 
  $\x_i$ & 
  $\ln \left( \f \left(\x_i\right) \right)$ &
  $\G_0 \left(\x_i\right)$ &
  $\G_1 \left(\x_i\right)$ &
  $\G_0^2 \left(\x_i\right)$ &
  $\G_1^2 \left(\x_i\right)$ &
  $\G_0 \left(\x_i\right) \G_1 \left(\x_i\right)$ &
  $\ln \left(\f \left(\x_i\right) \right) \G_0 \left(\x_i\right)$ &
  $\ln \left(\f \left(\x_i\right) \right) \G_1 \left(\x_i\right)$\\
\hline
$0$ & $1$ & $\ln 1,2$ & $1$ & $1$ & $1$ & $1$  & $1$ & $\ln 1,2$ & $\ln 1,2$\\
\hline
$1$ & $2$ & $\ln 0,9$ & $1$ & $2$ & $1$ & $4$  & $2$ & $\ln 0,9$ & $2 \ln 0,9$\\
\hline
$2$ & $3$ & $\ln 3,1$ & $1$ & $3$ & $1$ & $9$  & $3$ & $\ln 3,1$ & $3 \ln 3,1$\\
\hline
$3$ & $5$ & $\ln 6,1$ & $1$ & $5$ & $1$ & $25$ & $5$ & $\ln 6,1$ & $5 \ln 6,1$\\
\hline
& & & & $\sum$ & $4$ & $39$ & $11$ & $3,016$ & $12,405$  
\end{tabular}
  \item Obtemos o sistema:
\[
\begin{cases}
4 a_0 + 11 a_1 &= 3,016\\
%
11 a_0 + 39 a_1 &= 12,405
\end{cases}
\]
  \item Cuja solução é $a_0 = -0,538$ e $a_1 = 0,470$ e, portanto $a = e^{a_0} = 0,584$, $b = e^{a_1} = 1,600$ e $P\left(\x\right) = 0,584 \left(1,6\right)^\x$
\end{itemize}

\end{frame}