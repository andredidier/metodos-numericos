\section{Introdução}

\begin{frame}
\frametitle{Introdução}

\begin{itemize}
  \item Quando estudamos um fenômeno de forma experimental, é comum termos um conjunto de valores tabelados
  \item Utilizando tais informações podemos levantar várias questões
    \begin{itemize}
      \item Qual a relação existente entre $\x$ e $\f\left(x\right)$?
      \item Qual o valor de $\f\left(x\right)$ para um determinado $\x$ fora do tabelamento?
    \end{itemize}
\end{itemize}
\end{frame}

\begin{frame}
\frametitle{Introdução}

\begin{itemize}
  \item Nestas circustâncias, temos um tabelamento da forma:
%
\begin{center}
\onslide<2->{
\begin{tabular}{|c|c|c|c|c|}
\hline
$\x_i$ & $\x_0$ & $\x_1$ & \ldots & $\x_n$\\
\hline
$\f\left(\x_i\right)$ & $\f\left(\x_0\right)$ & $\f\left(\x_1\right)$ & \ldots & $\f\left(\x_n\right)$\\
\hline
\end{tabular}
}
\end{center}
%
  \item Como podemos usar o tabelamento para calcular o valor da função $\f$ desconhecida em pontos não tabelados?
  \item $\f\left(\x\right)$ mapeia algum fenômeno com dados colhidos de forma experimental
    \begin{itemize}
      \item<3-> Não temos certeza sobre corretude dos dados colhidos
    \end{itemize}
\end{itemize}

\end{frame}
