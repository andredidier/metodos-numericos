\section{Introdução}

\begin{frame}
\frametitle{Introdução}

\begin{itemize}
  \item Quando estudamos um fenômeno de forma experimental, é comum termos um conjunto de valores tabelados
  \item Utilizando tais informações podemos levantar várias questões
    \begin{itemize}
      \item Qual a relação existente entre $\x$ e $\f\left(x\right)$?
      \item Qual o valor de $\f\left(x\right)$ para um determinado $\x$ fora do tabelamento?
    \end{itemize}
\end{itemize}
\end{frame}

\begin{frame}
\frametitle{Introdução}

\begin{itemize}
  \item Nestas circustâncias, temos um tabelamento da forma:
%
\begin{center}
\onslide<2->{
\begin{tabular}{|c|c|c|c|c|}
\hline
$\x_i$ & $\x_0$ & $\x_1$ & \ldots & $\x_n$\\
\hline
$\f\left(\x_i\right)$ & $\f\left(\x_0\right)$ & $\f\left(\x_1\right)$ & \ldots & $\f\left(\x_n\right)$\\
\hline
\end{tabular}
}
\end{center}
%
  \item Como podemos usar o tabelamento para calcular o valor da função $\f$ desconhecida em pontos não tabelados?
  \item $\f\left(\x\right)$ mapeia algum fenômeno com dados colhidos de forma experimental
    \begin{itemize}
      \item<3-> Não temos certeza sobre corretude dos dados colhidos
    \end{itemize}
\end{itemize}

\end{frame}

\begin{frame}
\frametitle{Introdução}

\begin{itemize}[<+->]
  \item Previsão para o estoque de um determinado produto em função do histórico de sua demanda

  \item Previsão de inflação, consumo energético, dados polucionais etc.
\end{itemize}
\end{frame}

\begin{frame}
\frametitle{Ajustamento de curvas}

\begin{itemize}
  \item Definição: o problema de ajuste de curvas no caso em que temos um tabelamento de pontos $\left(\x_0, \f\left(\x_0\right)\right), \left(\x_1, \f\left(\x_1\right) \right), \ldots, \left(\x_n, \f\left(\x_n\right) \right) $, com $\x_0, \x_1, \ldots, \x_n \in [a,b]$, consiste em:
  
  \item Escolhidas $m+1$ funções $g_0\left(\x\right), g_1\left(\x\right), \ldots, g_m\left(\x\right) $, contínuas em $[a,b]$, obter $m+1$ constantes $a_0, a_1, \ldots, a_m$, tais que a função $P\left(\x \right) = a_0 g_0\left(\x\right) + a_1 g_1 \left(\x\right) + \ldots + a_m g_m \left(\x\right) $ se aproxime ao máximo de $\f\left(\x\right) $
\end{itemize}
\end{frame}

\begin{frame}
\frametitle{Ajustamento de curvas}

\begin{itemize}
  \item Temos uma combinação linear de funções elementares
  \item $P\left(\x\right) = \sum_{j=0}^m a_j \times G_j \left(x\right) $
    \begin{itemize}
      \item $a_j$: coeficientes a serem ajustados
      \item $G_j$: funções conhecidas ($1$, $\x$, $\sen x$, $\ln x$)
    \end{itemize}
  \item Desejamos escolher a função $P\left(\x\right)$ que melhor represente o tabelamento utilizado
\end{itemize}
\end{frame}

\begin{frame}
\frametitle{Ajustamento de curvas}

\begin{itemize}
  \item Como escolher as funções contínuas $g_0\left(\x\right), g_1\left(\x\right), \ldots, g_m\left(\x\right) $?
  \item Uma mandeira simples consiste em analisar os pontos conhecidos em um gráfico cartesiano
  \item Ex.:
    \begin{multicols}{2}
      \begin{itemize}
        \item $g_1\left(\x\right) = x^2$
        \item Procuramos o valor de a em: $\varphi \left(\x\right) = a\x^2$
        \item Ou seja, qual a parábola com equação $a\x^2$ melhor se ajusta aos dados?
        \item<2-> Esta escolha nem sempre é simples e não será objeto de estudo neste curso. 
      \end{itemize}    
      \columnbreak
      \begin{tikzpicture}[global scale=0.7]
      
      \draw plot[mark=*,only marks, xshift=3cm, yshift=3cm] file {ajustamento-dados1.txt};
      \draw[->] (0,3) -- coordinate (x axis mid) (6,3) node[right] {$\x$};
      \draw[->] (3,0) -- coordinate (y axis mid) (3,6) node[above] {$\f\left(\x\right) $};
      \foreach \x in {-1,0,1}
        \draw [xshift=3cm,yshift=3cm](\x,1pt) -- (\x,-3pt)
            node[anchor=north] {$\x$};
      \foreach \y/\ytext in {-2,-1,,1,2}
          \draw[xshift=3cm,yshift=3cm] (1pt,\y) -- (-3pt,\y) node[anchor=east] {$\ytext$};
          
      \end{tikzpicture}
    \end{multicols}
\end{itemize}

\end{frame}

\begin{frame}
\frametitle{Ajustamento de curvas}

\begin{itemize}
  \item O que significa obter uma curva que melhor se ajuste, ou que mais se aproxime de uma função $\f\left(\x\right)$ desconhecida?
  \item Ideia geométrica:\\
    \begin{tikzpicture}[global scale=0.7]
    
    %\draw plot[mark=*,only marks, xshift=0cm, yshift=0cm] file {ajustamento-dados2.txt};
    %\draw plot[xshift=0cm, yshift=0cm] file {ajustamento-dados3.txt};
    %\draw[->] (0,0) -- coordinate (x axis mid) (6,0) node[right] {$\x$};
    %\draw[->] (0,0) -- coordinate (y axis mid) (0,6) node[above] {$y $};
    \begin{axis}[%
      xmin=0, xmax=6, ymin=0, ymax=6, axis x line=center, axis y line=center
    ]
      \addplot plot[mark=*,only marks, xshift=0cm, yshift=0cm] file {ajustamento-dados2.txt};
      \addplot plot[mark=none, xshift=0cm, yshift=0cm] file {ajustamento-dados3.txt};
      
    \end{axis}

    \draw [decorate, decoration={brace, amplitude=2pt, mirror}, xshift=15pt, yshift=-3pt] (3,3.2) -- (3,3.7) node [black,midway,xshift=55pt] {%
    $R\left(\x_i\right) = P\left(\x_i\right) - \f\left(\x_i\right)$
    };
    
    \onslide<2->{\draw (5,5) node {Objetivo: tornar os resíduos $R\left(\x_i\right) mínimos$};}
    \end{tikzpicture}
\end{itemize}
\end{frame}

\begin{frame}
\frametitle{Ajustamento de curvas}

\begin{itemize}
  \item O que significa tornar os resíduos $R\left(\x_i\right)$ mínimos?
  \item $\sum_{i=0}^n R\left(\x_i\right) = 0$ ?
    \begin{itemize}
      \item Não! A curva pode ter resíduos positivos e negativos grandes em valores absolutos, mas que somados se aproximem bastante de zero. Escolha inadequada.
    \end{itemize}
  \item $\sum_{i=0}^n \left|R\left(\x_i\right)\right| = 0$ ?
    \begin{itemize}
      \item Não! Função valor absoluto não é derivável em seu mínimo. 
    \end{itemize}  
  \item $\sum_{i=0}^n R^2\left(\x_i\right) = 0$ ?
    \begin{itemize}
      \item Sim! Problemas anteriores são resolvidos
      \item Buscaremos a função do tipo escolhido que produza a menor soma dos quadrados dos resíduos
      \item Método dos mínimos quadrados (MMQ)
    \end{itemize}
\end{itemize}
\end{frame}

\begin{frame}
\frametitle{Método dos mínimos quadrados}

\begin{itemize}
  \item Função $\varphi$ associa a função escolhida para representar a tabela dada à soma dos quadrados dos resíduos produzidos por ela
  \item Procuramos o mínimo de\\
  %
  $\varphi\left(a_0,a_1,\ldots,a_m\right) = \displaystyle\sum_{i=0}^{n}R^2\left(\x_i\right)$
\end{itemize}
\end{frame}