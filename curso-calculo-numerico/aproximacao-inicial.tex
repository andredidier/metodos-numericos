\section{Aproximação inicial}

\begin{frame}
\frametitle{Isolamento de raízes}

\begin{itemize}
  \item Dada uma função $\f$ derivável em duas ordens e um intervalo $[a,b]$
  \item Se $\f\left(a\right) \f\left(b\right) > 0$, então existe um número par de raízes reais ou não existem raízes
  \item Se $\f\left(a\right) \f\left(b\right) < 0$, então existe um número ímpar de raízes reais
  \item Se $\f'\left(a\right) \f' \left(b\right) > 0$, então o comportamento da função no intervalo poderá ser apenas crescente ou apenas decrescente, nunca os dois se alternando
  \item Se $\f'\left(a\right) \f' \left(b\right) < 0$, a função alternará seu comportamento entre crescente e descrescente
  \item Se $\f''\left(a\right) \f'' \left(b\right) > 0$, então a concavidade não muda no intervalo
  \item Se $\f''\left(a\right) \f'' \left(b\right) < 0$, então a concavidade muda no intervalo
  \item Haverá uma única raiz se e somente se: $\f\left(a\right) \f \left(b\right) < 0$, $\f' \left(a\right) \f' \left(b\right) > 0$ e $\f'' \left(a\right) \f'' \left(b\right) > 0$
\end{itemize}
\end{frame}