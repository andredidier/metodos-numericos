%% LaTeX Beamer presentation template (requires beamer package)
%% see http://bitbucket.org/rivanvx/beamer/wiki/Home
%% idea contributed by H. Turgut Uyar
%% template based on a template by Till Tantau
%% this template is still evolving - it might differ in future releases!

\documentclass[portuges]{beamer}


\mode<presentation>
{
%\usetheme{CIn}
\usetheme{CIn20131120}

%\setbeamercovered{transparent}
}

\usepackage[portuges]{babel}
\usepackage[utf8]{inputenc}

% font definitions, try \usepackage{ae} instead of the following
% three lines if you don't like this look
\usepackage{mathptmx}
\usepackage[scaled=.90]{helvet}
\usepackage{courier}
\usepackage[T1]{fontenc}
\usepackage{hyperref}
\usepackage{cleveref}
\usepackage{amsmath}
\usepackage{pifont}
\usepackage{multicol}
\usepackage{pgfplots}
\usepackage{tikz}
\usepackage{icomma}
\usepackage{calculo-numerico}

\usetikzlibrary{arrows,trees,positioning,circuits.logic.US,shapes,decorations}

\DeclareGraphicsExtensions{.png,.eps}
%\DeclareGraphicsExtensions{.png}
\graphicspath{{./figures/},{./}}

\tikzset{
  >=stealth',
  edge from parent path={(\tikzparentnode.south) -- ++(0,-0.95cm)
      -| (\tikzchildnode.north)},
  global scale/.style={
    scale=#1,
    every node/.style={scale=#1}
  },
}

\makeatletter
\newcommand{\footnoteref}[1]{\protected@xdef\@thefnmark{\ref{#1}}\@footnotemark}
\makeatother

\def\x{x}
\def\f{\mathrm{f}}
\def\I{\mathrm{I}}
\def\k{\mathrm{k}}
\def\ln{\operatorname{\mathrm{ln}}}
\newcommand{\machinenumber}[2]{#1 \cdot 10^{#2}}
\newcommand{\machine}[4]{\mathrm{F}\left(#1,#2,#3,#4\right)}
\def\ok{\checkmark}
\def\alrd{Andr\'{e} L. R. Didier}
\DeclareMathOperator{\sen}{\text{sen}}
%\def\G{\mathcal{G}}
\def\G{g}
%\DeclareMathOperator{\cos}{\text{cos}}
%\DeclareMathOperator{\min}{\text{min}}
%\DeclareMathOperator{\max}{\text{max}}

\def\arraystretch{1.2}
\renewcommand{\tabcolsep}{1pt}

\addto\captionsportuges{
    % Second argument is singular, third is plural
    \crefname{equation}{eq.}{eqs.}
    \Crefname{equation}{Eq.}{Eqs.}
    \crefname{enumi}{item}{itens}
    \Crefname{enumi}{Item}{Itens}
}

\title{Métodos Numéricos}
\subtitle{}

%\subtitle{}

% - Use the \inst{?} command only if the authors have different
%   affiliation.
%\author{F.~Author\inst{1} \and S.~Another\inst{2}}
\author{%
Adalberto Júnior \and \\%
\alrd \and \\%
Rafael Mesquita \and \\%
Renato Vimieiro}

% - Use the \inst command only if there are several affiliations.
% - Keep it simple, no one is interested in your street address.
\institute[UFPE]
{
\inst{1}%
Departamento de Ciência da Computação\\
UFPE
}

% This is only inserted into the PDF information catalog. Can be left
% out.
\subject{Métodos Numéricos}



% If you have a file called "university-logo-filename.xxx", where xxx
% is a graphic format that can be processed by latex or pdflatex,
% resp., then you can add a logo as follows:

% \pgfdeclareimage[height=0.5cm]{university-logo}{university-logo-filename}
% \logo{\pgfuseimage{university-logo}}



% Delete this, if you do not want the table of contents to pop up at
% the beginning of each subsection:
%\AtBeginSubsection[]
%{
%\begin{frame}<beamer>
%\frametitle{Outline}
%\tableofcontents[currentsection,currentsubsection]
%\end{frame}
%}

\AtBeginLecture
{
\section{\insertlecture}
\title{\insertlecture}
\subtitle{\insertshortlecture}
\begin{frame}<beamer>
\titlepage
\end{frame}
}

\def\endlecture
{
\begin{frame}<beamer>{Continue estudando!}
\begin{center}
\href{http://www.cin.ufpe.br/~if215}{\includeautosizegraphics{if215}}
\end{center}
\end{frame}
}

% If you wish to uncover everything in a step-wise fashion, uncomment
% the following command:

%\beamerdefaultoverlayspecification{<+->}

%NÃO DESCOMENTE A LINHA ABAIXO NO BRANCH MASTER. PARA USAR UMA AULA 
%ESPECÍFICA, UTILIZE O BRANCH DA AULA. SE NÃO ESTIVER CRIADO AINDA, CRIE.

%\includeonlylecture{aula01}
%\includeonlylecture{aula01}
%\includeonlylecture{aula02}
%\includeonlylecture{aula03}
%\includeonlylecture{aula04}
%\includeonlylecture{aula05}
%\includeonlylecture{aula06}
%\includeonlylecture{aula07}
%\includeonlylecture{aula08}
%\includeonlylecture{aula09}
%\includeonlylecture{aula10}
%\includeonlylecture{aula11}
%\includeonlylecture{aula12}
%\includeonlylecture{aula13}
\includeonlylecture{aula14}

\begin{document}

\lecture[Aula 1]{Apresentação da disciplina}{aula01}
\endlecture

\lecture[Aula 2]{Bases, erros e aritmética de ponto flutuante}{aula02}
\endlecture

\lecture[Aula 3]{Arredondamento e operações aritméticas}{aula03}
\endlecture

\lecture[Aula 4]{Aproximação inicial, método da bisseção e método das cordas}{aula04}
\section{Aproximação inicial}

\begin{frame}
\frametitle{Isolamento de raízes}

\begin{itemize}
  \item Dada uma função $\f$ derivável em duas ordens e um intervalo $[a,b]$
  \item Se $\f\left(a\right) \f\left(b\right) > 0$, então existe um número par de raízes reais ou não existem raízes
  \item Se $\f\left(a\right) \f\left(b\right) < 0$, então existe um número ímpar de raízes reais
  \item Se $\f'\left(a\right) \f' \left(b\right) > 0$, então o comportamento da função no intervalo poderá ser apenas crescente ou apenas decrescente, nunca os dois se alternando
  \item Se $\f'\left(a\right) \f' \left(b\right) < 0$, a função alternará seu comportamento entre crescente e descrescente
  \item Se $\f''\left(a\right) \f'' \left(b\right) > 0$, então a concavidade não muda no intervalo
  \item Se $\f''\left(a\right) \f'' \left(b\right) < 0$, então a concavidade muda no intervalo
  \item Haverá uma única raiz se e somente se: $\f\left(a\right) \f \left(b\right) < 0$, $\f' \left(a\right) \f' \left(b\right) > 0$ e $\f'' \left(a\right) \f'' \left(b\right) > 0$
\end{itemize}
\end{frame}
\endlecture

\lecture[Aula 5]{Método iterativo linear e método de Newton}{aula05}
\endlecture

\lecture[Aula 6]{Método das secantes e critérios de parada}{aula06}
\endlecture

\lecture[Aula 7]{Exercícios sobre zeros de funções}{aula07}
\section {Introdução}

\begin{frame}
\frametitle{Introdução}
Todas as questões foram obtidas da 3ª edição do livro ``Métodos Numéricos'' de
José Dias dos Santos e Zanoni Carvalho da Silva.
\end{frame}

\section{Noções de Aritmética de Máquina}
\subsection[Ponto flutuante]{Ponto flutuante, aritmética e arredondamento}

\subsubsection{Questão 1.6}

\begin{frame}
\frametitle{Questão 1.6}
\framesubtitle{Representação numérica, aritmética de ponto flutuante,
arredondamento}

Considere a máquina $\machine{10}{5}{-9}{9}$. Nela, verifique se 
$(a+b)+c = a + (b+c)$, onde
\begin{align*}
a &= 32.424\\
b &= 4.2131\\
c &= 0.000382
\end{align*} 

\end{frame}

\begin{frame}
\frametitle{Questão 1.6}
\framesubtitle{$\machine{10}{5}{-9}{9}$}

Normalizando os números, temos:
\begin{align*}
a = 32.424 &= \machinenumber{3.2424}{1}\\
b = 4.2131 &= \machinenumber{4.2131}{0}\\
c = 0.000382 &= \machinenumber{3.8200}{-4}
\end{align*} 


\end{frame}

\begin{frame}
\frametitle{Questão 1.6}
\framesubtitle{$\machine{10}{5}{-9}{9}$, 
$a = \machinenumber{3.4240}{1}$, 
$b = \machinenumber{4.2131}{0}$,
$c = \machinenumber{3.8200}{-4}$}

Fazendo $(a+b)+c$, temos:
{\tiny
\begin{align*}
& (\machinenumber{3.2424}{1} + \machinenumber{4.2131}{0}) +
\machinenumber{3.8200}{-4} \\
%
= &(\machinenumber{3.2424}{1} + \machinenumber{0.42131}{1}) +
\machinenumber{3.8200}{-4}\\
%
= &\machinenumber{3.6637}{1} +\machinenumber{3.8200}{-4}\\
%
= &\machinenumber{3.6637}{1} +\machinenumber{0.0000382}{1}\\
%
= &\machinenumber{3.6637}{1}\\
\end{align*}
}

E fazendo $a + (b+c)$, temos:
{\tiny
\begin{align*}
& \machinenumber{3.2424}{1} + 
  (\machinenumber{4.2131}{0} + \machinenumber{3.8200}{-4})\\
= &\machinenumber{3.2424}{1} + 
  (\machinenumber{4.2131}{0} + \machinenumber{0.000382}{0})\\
= &\machinenumber{3.2424}{1} + \alert<2->{\machinenumber{4.2135}{0}}
\onslide<2->{\text{ [arredondamento]}}\\
= &\machinenumber{3.2424}{1} + \machinenumber{0.42135}{1}\\
= &\alert<2->{\machinenumber{3.6638}{1}}\onslide<2->{\text{ [arredondamento]}}
\end{align*}
}
\onslide<2->{Os resultados são diferentes.}

\end{frame}

\subsubsection{Questão 1.7}

\begin{frame}
\frametitle{Questão 1.7}
\framesubtitle{Representação numérica, aritmética de ponto flutuante}

Considere um computador hipotético que trabalha na base 10, com 5 dígitos no
significando e 2 dígitos no expoente, denotado por $\machine{10}{5}{-99}{99}$.
Nele calcue o valor de:
\[
S = \sum_{n=0}^{4} \frac{1}{7^n}
\]
de duas formas: (i) da maior parcela para a menor e (ii) da menor parcela para a
maior.

O que dizer\footnote{Alguma propriedade dos números reais não foi verificada?
Dos dois resultados, qual o mais próximo do verdadeiro? Etc.} diante dos
resultados dos itens (i) e (ii)?

\end{frame}

\begin{frame}
\frametitle{Questão 1.7}
\framesubtitle{$\machine{10}{5}{-99}{99}$}

Em (i), calculamos: $\left(\left(\left(\frac{1}{7^0} + \frac{1}{7^1}\right) +
\frac{1}{7^2}\right) + \frac{1}{7^3}\right) + \frac{1}{7^4}$.
%
Temos:
{\tiny
\begin{align*}
  & 
  \left(
    \left(
      \left(\machinenumber{1.0000}{0} + \machinenumber{1.4286}{-1}\right) +
      \machinenumber{2.0408}{-2}
    \right) +
    \machinenumber{2.9155}{-3}
  \right) +
  \machinenumber{4.1649}{-4}\\
= &
  \left(
    \left(
      \machinenumber{1.1429}{0} + \machinenumber{2.0408}{-2}
    \right) +
    \machinenumber{2.9155}{-3}
  \right) +
  \machinenumber{4.1649}{-4}\\
= &
  \left(
      \machinenumber{1.1633}{0} + \machinenumber{2.9155}{-3}
  \right) +
  \machinenumber{4.1649}{-4}\\
= & \machinenumber{1.1662}{0} + \machinenumber{4.1649}{-4}\\
= & \alert<2->{\machinenumber{1.1666}{0}}
\end{align*}
}

Em (ii), calculamos: $\frac{1}{7^0} + \left(\frac{1}{7^1} +
\left(\frac{1}{7^2} + \left(\frac{1}{7^3} + \frac{1}{7^4}\right)\right)\right)$.
%
Temos:
{\tiny
\begin{align*}
  & 
  \machinenumber{1.0000}{0} + \left(\machinenumber{1.4286}{-1} +
  \left(\machinenumber{2.0408}{-2} + \left(\machinenumber{2.9155}{-3} +
  \machinenumber{4.1649}{-4}\right)\right)\right)\\
%
  =& 
  \machinenumber{1.0000}{0} + 
  \left(\machinenumber{1.4286}{-1} +
    \left(\machinenumber{2.0408}{-2} + 
      \machinenumber{3.3319}{-3}
    \right)
  \right)\\
  =& 
  \machinenumber{1.0000}{0} + 
  \left(\machinenumber{1.4286}{-1} +
    \machinenumber{2.3740}{-2}  
  \right)\\
  =& \machinenumber{1.0000}{0} + \machinenumber{1.6660}{-1} \\
  =& \alert<2->{\machinenumber{1.1666}{0}}
\end{align*}
}

\end{frame}

\subsubsection{Questão 1.11.e}

\begin{frame}
\frametitle{Questão 1.11.e}
\framesubtitle{Representação numérica, aritmética de ponto flutuante}

Considere o sistema de ponto flutuante dado por $\machine{10}{6}{-99}{99}$. Os
elementos $x = \machinenumber{0.4721025}{8}$, $y = \machinenumber{1.00321}{5}$ e
$z = \machinenumber{0.0072134}{6}$ pertencem a essa máquina. Verifique usando as
representações de $x$, $y$ e $z$ neste sistema, se $x \cdot (y + z) = x \cdot y
+ x \cdot z$.

\end{frame}

\begin{frame}
\frametitle{Questão 1.11.e}
\framesubtitle{$\machine{10}{6}{-99}{99}$, $x = \machinenumber{0.4721025}{8} $,
$y = \machinenumber{1.00321}{5} $, $z = \machinenumber{0.0072134}{6} $}

Normalizando os números, temos:
\begin{align*}
x = \machinenumber{0.4721025}{8} =& \alert<2>{\machinenumber{4.72102}{7}}
\onslide<2->{\text{ [arredondamento]}}\\
y = \machinenumber{1.00321}{5} =& \machinenumber{1.00321}{5}\\
z = \machinenumber{0.0072134}{6} =& \machinenumber{7.21340}{3}\\
\end{align*}

\end{frame}

\begin{frame}
\frametitle{Questão 1.11.e}
\framesubtitle{$\machine{10}{6}{-99}{99}$, $x = \machinenumber{4.72102}{7} $,
$y = \machinenumber{1.00321}{5} $, $z = \machinenumber{7.21340}{3} $}

Fazendo $x \cdot (y + z)$, temos:
\begin{align*}
& \machinenumber{4.72102}{7} \cdot \left(\machinenumber{1.00321}{5} +
\machinenumber{7.21340}{3} \right) \\
= & \machinenumber{4.72102}{7} \cdot \machinenumber{1.07534}{5} \\
= & \alert<2->{\machinenumber{5.07670}{12}}
\end{align*}

Fazendo $x \cdot y + x \cdot z$, temos:
\begin{align*}
& \left(\machinenumber{4.72102}{7} \cdot \machinenumber{1.00321}{5}\right) 
  +
  \left(\machinenumber{4.72102}{7} \cdot \machinenumber{7.21340}{3} \right) \\
= &  \machinenumber{4.73617}{12} + \machinenumber{3.40546}{11}\\
= & \machinenumber{4.73617}{12} + \machinenumber{0.340546}{12}\\
= & \alert<2->{\machinenumber{5.07672}{12}}
\end{align*}

\onslide<2->{Logo, os cálculos têm resultados diferentes.}

\end{frame}

\section{Zeros de funções}
\subsection[Métodos de quebra e métodos iterativos]{Bisseção, falsa posição
(cordas), M.I.L., Newton e secantes}

\subsubsection{Questão 2.1}

\begin{frame}
\frametitle{Questão 2.1}
\framesubtitle{Bisseção, falsa posição (cordas), M.I.L., Newton e secantes}
{\small
Para cada função:
\begin{enumerate}
  \item Localizar, se existir, raiz real mais próxima da origem;
  \item Determinar analiticamente um intervalo de amplitude $0.1$ contendo tal
  raiz;
  \item Aplicar os métodos abaixo para calcular a raiz aproximada: 
  \begin{enumerate}{\footnotesize
    \item Bisseção\footnote{Para o método da Bisseção faça até que o intervalo
    de separação seja menor que $10^{-2}$ e indique quantas iterações serão
    necessárias antes de aplicar o método. Para os demais métodos, faça até que
    $|x_{i+1} - x_i| \leq 10^{-3}$ ou $i = 3$}
    \item Falsa posição (cordas)
    \item Iterativo linear
    \item Newton-Raphson
    \item Das secantes}
  \end{enumerate}
\end{enumerate}
Considere uma máquina com $5$ casas decimais e arredondamento padrão.
}

\end{frame}

\begin{frame}
\frametitle{Questão 2.1.g - localizar raiz real próxima à origem}
\framesubtitle{$\f\left(\x\right) = \x \cdot e^\x + \x^2 - 1$}

Vamos verificar alguns valores próximos da origem que facilitem os cálculos para
verificar a mudança de sinal:
\begin{center}
\begin{tabular}{r|r}
$\x$ & $\f\left(\x\right)$ \\
\hline
\hline
$\onslide<2->{-1}$ & $\onslide<3->{-\frac{1}{e}}$\\
\hline
$\onslide<4->{0}$ & $\onslide<5->{-1.00000}$\\
\hline
$\onslide<6->{1}$ & $\onslide<7->{e}$ \\
\hline
\end{tabular}
\end{center}

\onslide<7->{Há mudança de sinal no intervalo $[0,1]$, logo existe ao menos uma
raiz real.}
\end{frame}

\begin{frame}
\frametitle{Questão 2.1.g - encontrar o intervalo de separação de amplitude
$0.1$}
\framesubtitle{$\f\left(\x\right) = \x \cdot e^\x + \x^2 - 1$}

Como $\f$ é uma função crescente\footnote{$\f'\left(\x\right) =
\left(\x + 1\right)e^\x + 2 \cdot \x$} no intervalo $[0,1]$, calculamos os
valores de $\f$ com incremento de $0.1$ a partir de 0:
{\scriptsize
\begin{center}
\begin{tabular}{r|r}
$\x$ & $\f\left(\x\right)$ \\
\hline
\hline
$0$ & $-1.00000$\\
\hline
\onslide<2->{$0.1$} & \onslide<3->{$-8.79482908 \cdot 10^{-1}$}\\
\hline
\onslide<4->{$0.2$} & \onslide<5->{$-7.15719448 \cdot 10^{-1}$}\\
\hline
\onslide<6->{$0.3$} & \onslide<7->{$-5.05042358 \cdot 10^{-1}$}\\
\hline
\onslide<8->{$0.4$} & \onslide<9->{$-2.43270121 \cdot 10^{-1}$}\\
\hline
\onslide<10->{$0.5$} & \onslide<11->{$7.4360635 \cdot 10^{-2}$}\\
\hline
\end{tabular}
\end{center}
}

\onslide<11->{Logo, o intervalo de separação de amplitude $0.1$ é $[0.4,0.5]$.}
\end{frame}

\begin{frame}
\frametitle{Questão 2.1.g - calcular a quantidade de iterações no método da
bisseção}
\framesubtitle{$\f\left(\x\right) = \x \cdot e^\x + \x^2 - 1$, 
$\f'\left(\x\right) = \left(\x + 1\right)e^\x + 2 \cdot \x$, $\I =
[0.4,0.5]$, $l = 10^{-2}$}

A quantidade de iterações é dada pela fórmula:
\[
\k \geq \frac{\ln\left(b_0 - a_0\right) - \ln l}{\ln 2}
\]

Substituindo pelos valores:
\begin{align*}
\k &\geq \frac{\ln\left(\onslide<2->{0.5} - \onslide<3->{0.4}\right) - \ln
\onslide<4->{10^{-2}}}{\ln 2}\\
\k &\geq \onslide<5->{3.32193}
\end{align*}

\onslide<5->{A quantidade de iterações necessárias é $4$.}

\end{frame}

\begin{frame}
\frametitle{Questão 1.g - aplicar o método da bisseção }
\framesubtitle{$\f\left(\x\right) = \x \cdot e^\x + \x^2 - 1$,
$\f'\left(\x\right) = \left(\x + 1\right)e^\x + 2 \cdot \x$ 
$\I = [0.4,0.5]$, $l = 10^{-2}$, $t = 4$, $i_{max} = 3$}

\[
\x = \frac{a + b}{2}
\]

{\scriptsize
\begin{center}
\begin{tabular}{r|r|r|r|r|r|r}
 & $a$ & $b$ & $\f\left(a\right)$ & $\f\left(b\right)$ & $\x$ &
$\f\left(\x\right)$ \\
\hline
\hline
$0$ & 
  $0.4$ & 
  \alert<3>{$0.5$} &
  \onslide<2->{$\machinenumber{-2.43270}{-1}$} &
  \onslide<2->{\alert<3>{$\machinenumber{7.43606}{-2}$}} & 
  \alert<3>{$0.45$} &
  \onslide<2->{\alert<3>{$\machinenumber{-9.17595}{-2}$}}
\\
\hline
\onslide<4->{$1$} & 
  \onslide<4->{$0.45$} &
  \onslide<4->{\alert<6>{$0.5$}} & 
  \onslide<5->{$\machinenumber{-9.17595}{-2}$} &
  \onslide<5->{\alert<6>{$\machinenumber{7.43606}{-2}$}} & 
  \onslide<4->{\alert<6>{$0.475$}} &
  \onslide<5->{\alert<6>{$\machinenumber{-1.05683}{-2}$}}
\\
\hline
\onslide<6->{$2$} & 
  \onslide<6->{\alert<8>{$0.475$}} &
  \onslide<6->{$0.5$} &  
  \onslide<7->{\alert<8>{$\machinenumber{-1.05683}{-2}$}} &
  \onslide<7->{$\machinenumber{7.43606}{-2}$} &
  \onslide<6->{\alert<8>{$0.4875$}} &
  \onslide<7->{\alert<8>{$\machinenumber{3.14235}{-2}$}}
\\
\hline
\onslide<8->{$3$} & 
  \onslide<8->{\alert<10>{$0.475$}} &
  \onslide<8->{$0.4875$} &  
  \onslide<9->{\alert<10>{$\machinenumber{-1.05683}{-2}$}} &
  \onslide<9->{$\machinenumber{3.14235}{-2}$} &
  \onslide<8->{\alert<10>{$0.48125$}} &
  \onslide<9->{\alert<10>{$\machinenumber{1.03101}{-2}$}}
\\
\hline
\onslide<10->{$4$} & 
  \onslide<10->{$0.475$} &
  \onslide<10->{$0.48125$} & 
  \onslide<11->{$\machinenumber{-1.05683}{-2}$} &
  \onslide<11->{$\machinenumber{1.03101}{-2}$} &
  \onslide<10->{$0.478125$} & 
  \onslide<11->{$\machinenumber{-1.58339}{-4}$}
\\
\hline
\end{tabular}
\end{center}
}

\onslide<12->{$b_4 - a_4 = 0.00625 < 10^{-2}$ e $\x = 0.478125$}
\end{frame}

\begin{frame}
\frametitle{Questão 2.1.g - aplicar o método das cordas }
\framesubtitle{$\f\left(\x\right) = \x \cdot e^\x + \x^2 - 1$,
$\f'\left(\x\right) = \left(\x + 1\right)e^\x + 2 \cdot \x$ 
$\I = [0.4,0.5]$, $|\x_{i+1} - \x_i| \leq 10^{-3}$, $i_{max} = 3$}

\[
\x = \frac{a \cdot \f\left(b\right) - b \cdot
\f\left(a\right)}{\f\left(b\right) - \f\left(a\right)}
\]

{\tiny
\begin{center}
\begin{tabular}{r|r|r|r|r|r|r|r}
  & 
  $a$ & 
  $b$ &
  $\f\left(a\right)$ &
  $\f\left(b\right)$ &
  $\x_i$ &
  $\f\left(\x_i\right)$ &
  $|\x_{i+1} - \x_i|$
\\
\hline
\hline
  1 &
  $\machinenumber{4.00000}{-1}$ &
  $\machinenumber{5.00000}{-1}$ &
  $\machinenumber{-2.43270}{-1}$&
  $\machinenumber{7.43610}{-2}$&
  $\machinenumber{4.76589}{-1}$ &
  $\machinenumber{-5.28300}{-3}$&
  ---\\
\hline
  2 &
  $\machinenumber{4.76589}{-1}$&
  $\machinenumber{5.00000}{-1}$ &
  $\machinenumber{-5.28300}{-3}$ &
  $\machinenumber{7.43610}{-2}$ &
  $\machinenumber{4.78142}{-1}$ &
  $\machinenumber{-1.02000}{-4}$ &
  $\machinenumber{1.55292}{-3}$\\
\hline
  3 &
  $\machinenumber{4.78142}{-1}$&
  $\machinenumber{5.00000}{-1}$ &
  $\machinenumber{-1.02000}{-4}$ &
  $\machinenumber{7.43610}{-2}$ &
  $\machinenumber{4.78172}{-1}$ &
  $\machinenumber{-2.00000}{-6}$ &
  $\machinenumber{2.99415}{-5}$\\
\hline
\end{tabular}
\end{center}
}

Encontramos $\x = 0.478172$.

\end{frame}

\begin{frame}
\frametitle{Questão 2.1.g - aplicar o M.I.L. }
\framesubtitle{$\f\left(\x\right) = \x \cdot e^\x + \x^2 - 1$,
$\f'\left(\x\right) = \left(\x + 1\right)e^\x + 2 \cdot \x$ 
$\I = [0.4,0.5]$, $|\x_{i+1} - \x_i| \leq 10^{-3}$, $i_{max} = 3$}

{\footnotesize
Inicialmente precisamos determinar uma função geradora $\varphi$. Como $\x
\cdot e^\x + \x^2 - 1 = 0$, podemos escolher 
$\varphi_1\left(\x\right) = x = \frac{1 - \x^2}{e^\x}$. Sua derivada é: 
$\varphi_1'\left(\x\right) = 
    \frac{\left(x^2-1\right)\cdot e^\x - 2\x \cdot e^\x}
  {e^{2x}}$

Vamos verificar as condições de convergência:
\begin{enumerate}
  \item As duas funções $\varphi_1$ e $\varphi_1'$ são contínuas.
  \item $|\varphi_1'\left(\x\right)| \leq k < 1, \forall x \in \I$:\\
\begin{tabular}{r|r}
$\x$ & $|\varphi_1'\left(\x\right) |$ \\
\hline
\hline
$\machinenumber{4.0000}{-1}$ & $\machinenumber{1.09932}{0}$ \\
\hline
$\machinenumber{4.5000}{-1}$ & $\machinenumber{1.08237}{0}$ \\
\hline
$\machinenumber{5.0000}{-1}$ & $\machinenumber{1.06143}{0}$ \\
\hline
\end{tabular}
\end{enumerate}

Logo $\varphi_1'$ não pode ser usada. 

Trivialmente não podemos escolher. Tente obter $\varphi_2$ e $\varphi_3$
isolando os outros termos da equação.

}
\end{frame}

\begin{frame}
\frametitle{Questão 2.1.g - aplicar o método new Newton-Raphson}
\framesubtitle{$\f\left(\x\right) = \x \cdot e^\x + \x^2 - 1$,
$\f'\left(\x\right) = \left(\x + 1\right)e^\x + 2 \cdot \x$ 
$\I = [0.4,0.5]$, $|\x_{i+1} - \x_i| \leq 10^{-3}$, $i_{max} = 3$}

A função geradora é definida por: 
\[
\varphi\left(\x_i\right) = \x_{i+1} = \x_i -
\frac{\f\left(\x_i\right)}{\f'\left(\x_i\right)}
\]

Logo, temos que:
 
\[
\varphi\left(\x_i\right) = \x_{i+1} = \x_i -
\frac{\x_i \cdot e^\x_i + \x_i^2 - 1}
{\left(\x_i + 1\right)e^\x_i + 2 \cdot \x_i}
\]

Aplicando o método obtemos:\\
{\scriptsize
\begin{center}
\begin{tabular}{r|r|r|r}
  &
  $\x$ &
  $\varphi\left(\x\right)$ &
  $|\x_{i+1} - \x_i|$\\
\hline
\hline
0 & $\machinenumber{4.50000}{-1}$ & $\machinenumber{4.78909}{-1}$ & ---\\
\hline
1 & $\machinenumber{4.78909}{-1}$ & $\machinenumber{4.78173}{-1}$ 
  & $\machinenumber{7.36000}{-4}$\\
\hline
\end{tabular}
\end{center}
}

Encontramos $\x = 0.478173$.

\end{frame}

\begin{frame}
\frametitle{Questão 2.1.g - aplicar o método das secantes}
\framesubtitle{$\f\left(\x\right) = \x \cdot e^\x + \x^2 - 1$,
$\f'\left(\x\right) = \left(\x + 1\right)e^\x + 2 \cdot \x$ 
$\I = [0.4,0.5]$, $|\x_{i+1} - \x_i| \leq 10^{-3}$, $i_{max} = 3$}

A função geradora é definida por: 
\[
\varphi\left(\x_i\right) = \x_{i+1} = 
\frac
  {\x_{i-1} \cdot \f\left(\x_i\right) - \x_i \cdot \f \left(\x_{i-1}\right)}
  {\f\left(\x_i\right) - \f\left(\x_{i-1}\right)}
\]

Aplicando o método obtemos:\\
{\tiny
\begin{center}
\begin{tabular}{r|r|r|r|r|r|r}
  &
  $\x_i$ &
  $\x_{i-1} $ &
  $\f\left(\x_i\right) $ &
  $\f\left(\x_{i-1}\right) $ &
  $\varphi\left(\x_i\right) $ &
  $|\x_{i+1} - \x_i|$ \\
\hline
\hline
  0 &
  $\machinenumber{4.00000}{-1}$ &
  $\machinenumber{5.00000}{-1}$ &
  $\machinenumber{-2.43270}{-1}$ &
  $\machinenumber{7.43610}{-2}$ &
  $\machinenumber{4.76589}{-1}$ &
  ---
  \\
\hline
  1 &
  $\machinenumber{4.76589}{-1}$ &
  $\machinenumber{4.00000}{-1}$ &
  $\machinenumber{-5.28000}{-3}$ &
  $\machinenumber{-2.43270}{-1}$ &
  $\machinenumber{4.78289}{-1}$ &
  $\machinenumber{1.70000}{-3}$
  \\
\hline
  2 &
  $\machinenumber{4.78289}{-1}$ &
  $\machinenumber{4.76589}{-1}$ &
  $\machinenumber{3.90000}{-4}$ &
  $\machinenumber{-5.28200}{-3}$ &
  $\machinenumber{4.78172}{-1}$ &
  $\machinenumber{1.17000}{-4}$
  \\
\hline
\end{tabular}
\end{center}
}

Encontramos $\x = 0.478172$.
\end{frame}


\endlecture

\lecture[Aula 8]{Sistema de equações lineares e método de eliminação de Gauss}{aula08}
\endlecture

\lecture[Aula 12]{Exercícios sobre sistemas de equações lineares}{aula12}

\begin{frame}
\frametitle{Introdução}
Todas as questões foram obtidas da 3ª edição do livro ``Métodos Numéricos'' de
José Dias dos Santos e Zanoni Carvalho da Silva.
\end{frame}

\begin{frame}
\frametitle{Questão 3.1}
Considere o sistema de equações lineares
\[\left\{
\begin{array}{llll}
5x_1 &+ 6x_2 &+ 12x_3 &= -1 \\
2x_1 &- 3x_2 &&= -1\\
4x_1 &+x_2 &-x_3 &= 6
\end{array}
\right.\]

Resolva-o usando os seguintes métodos (considere 4 casas decimais e o arredondamento padrão):
\begin{enumerate}
  \item Eliminação de Gauss;
  \item Eliminação de Gauss-Jordan;
  \item Decomposição LU.
\end{enumerate}
\end{frame}

\begin{frame}
\frametitle{Questão 3.1 pelo método de eliminação de Gauss}
\small
Pelo método de eliminação de Gauss, devemos transformar $Ax = b$ em $T x = c$ onde $T$ é uma matriz triangular.
%
Para encontrar $T$ trocamos linhas ou colunas e multiplicamos por constantes.
%
Para a matriz extendida do problema:
\[A = 
\begin{pmatrix}
5 & 6 & 12 & | & -1\\
2 & -3 & 0 & | & -1\\
4 & 1 & -1 & | & 6 
\end{pmatrix}
\]
Zeramos os elementos da primeira coluna usando
$m_{11} = -\frac{a_{21}}{a_{11}} = - \frac{2}{5} = -0.4$ e $m_{12}=-\frac{a31}{a11} = -\frac{4}{5}=-0.8$:
\[
\begin{pmatrix}
5 & 6 & 12 & | & -1\\
0 & -5.4 & -4.8 & | & -0.6\\
0 & -3.8 & -10.6 & | & 6.8 
\end{pmatrix}
\]
\end{frame}

\begin{frame}
\frametitle{Questão 3.1 pelo método de eliminação de Gauss}
\small
A partir da nova matriz
\[
\begin{pmatrix}
5 & 6 & 12 & | & -1\\
0 & -5.4 & -4.8 & | & -0.6\\
0 & -3.8 & -10.6 & | & 6.8 
\end{pmatrix}
\]
Zeramos os elementos da segunda coluna usando $m_{21} = -\frac{\dot{a}_{32}}{\dot{a}_{22}} = -\frac{-3.8}{-5.4}=-0.70370$:
\[
\begin{pmatrix}
5 & 6 & 12 & | & -1\\
0 & -5.4 & -4.8 & | & -0.6\\
0 & 0 & -7.2222 & | & 7.2222 
\end{pmatrix}
\]
\end{frame}

\begin{frame}
\frametitle{Questão 3.1 pelo método de eliminação de Gauss}
\small
Da fórmula geral
\[
\left|
\begin{array}{ll}
x_n &= \frac{b_n}{a_{nn}}\\
x_i &= \frac{b_i - \sum_{j=i+1}^{n}a_{ij}x_j}{a_{ii}} \text{, }i = (n-1), (n-2), \ldots, 1
\end{array} 
\right.
\]
Temos:\\
$
\left|
\begin{array}{ll}
x_3 &= \frac{b_3}{a_{33}}\\
x_2 & = \frac{b_2 - a_{23}x_3}{a_{22}}\\
x_1 & = \frac{b_1 - a_{12}x_2 - a_{13}x_3}{a_{11}}
\end{array} 
\right.
$ e a matriz
$
\begin{pmatrix}
5 & 6 & 12 & | & -1\\
0 & -5.4 & -4.8 & | & -0.6\\
0 & 0 & -7.2222 & | & 7.2222 
\end{pmatrix}
$

Logo:
\[
\left|
\begin{array}{lll}
x_3 &= -\frac{7.2222}{7.2222}&=-1\\
x_2 & = \frac{-0,6 - (-4.8 \times (-1))}{-5.4} &=1\\
x_1 & = \frac{-1 - 6\times 1 - 12 \times (-1)}{5} &=1
\end{array} 
\right.
\]

\end{frame}

\begin{frame}
\frametitle{Questão 3.1 pelo método de eliminação de Gauss-Jordan}
\small
Neste método, transformamos a matriz $A$ em uma matriz diagonal. Vamos continuar a transformação do método de Gauss:
\[
\begin{pmatrix}
5 & 6 & 12 & | & -1\\
0 & -5.4 & -4.8 & | & -0.6\\
0 & 0 & -7.2222 & | & 7.2222 
\end{pmatrix}
\]
Zeramos os elementos da terceira coluna usando $m_{31}=-\frac{a_{23}}{a_{33}}=-\frac{-4.8}{-7.2222}=-0.66461$ e $m_{32}=-\frac{a_{13}}{a_{33}}=-\frac{12}{-7.2222}=-1.6615$:
\[
\begin{pmatrix}
5 & 6 & 0 & | & 11\\
0 & -5.4 & 0 & | & -5.3999\\
0 & 0 & -7.2222 & | & 7.2222 
\end{pmatrix}
\]
\end{frame}

\begin{frame}
\frametitle{Questão 3.1 pelo método de eliminação de Gauss-Jordan}
Da matriz
\[
\begin{pmatrix}
5 & 6 & 0 & | & 11\\
0 & -5.4 & 0 & | & -5.3999\\
0 & 0 & -7.2222 & | & 7.2222 
\end{pmatrix}
\]
Zeramos os elementos da segunda coluna usando $m_{22}=-\frac{a_{12}}{a_{22}}=-\frac{6}{-5.4}=1.1111$:
\[
\begin{pmatrix}
5 & 0 & 0 & | & 5\\
0 & -5.4 & 0 & | & -5.3999\\
0 & 0 & -7.2222 & | & 7.2222 
\end{pmatrix}
\]
%
Temos $x_1=\frac{5}{5}=1$, $x_2=\frac{-5.3999}{-5.4}=\machinenumber{9.9998}{-1}$ e $x_3=\frac{7.2222}{-7.2222}=-1$
\end{frame}


\begin{frame}
\frametitle{Questão 3.1 pelo método de decomposição LU}
No método da decomposição LU, transformamos o problema $Ax = b$ em $Ux = y$ e $Ly = b$, onde $A = LU$
\[A = 
\begin{pmatrix}
5 & 6 & 12 & | & -1\\
2 & -3 & 0 & | & -1\\
4 & 1 & -1 & | & 6 
\end{pmatrix}
\]

Antes de obter $L$ e $U$, precisamos verificar se $\det(A(1:k,1:k)) \ne 0$ ($k=1,2,\ldots,n$):

$
\begin{array}{c}
\begin{vmatrix}
5
\end{vmatrix} \ne 0\\
%
\onslide<2>{\text{\ok}}
\end{array}
$,%
%
$
\begin{array}{c}
\begin{vmatrix}
5 & 6 \\
2 & 3
\end{vmatrix} \ne 0\\
%
\onslide<2>{\text{\ok}}
\end{array}
$
e
$
\begin{array}{c}
\begin{vmatrix}
5 &6 &12\\
2 &-3 &0\\
4 &1 &-1
\end{vmatrix} \ne 0\\
%
\onslide<2>{\text{\ok}}
\end{array}
$

\end{frame}

\begin{frame}
\frametitle{Questão 3.1 pelo método de decomposição LU}
Dadas as fórmulas gerais para $L$ e $U$:
\[
\left\{
\begin{array}{lll}
l_{ij} = & a_{ij} - \sum_{k=1}^{j-1}l_{ik}u_{kj}, & i \ge  j\\
%
u_{ij} = & \frac{a_{ij}-\sum_{k=1}^{i-1}l_{ik}u_{kj} }{l_{ii}}, & i < j
\end{array}
\right.
\]
Calculamos:
$
L = 
\begin{pmatrix}
l_{11}=a_{11} &0 &0 \\
l_{21}=a_{21} &l_{22}=a_{22}-l_{21}u_{12} &0 \\
l_{31}=a_{31} &l_{32}=a_{32}-l_{31}u_{12} &l_{33}=a_{33} -l_{31}u_{13} -l_{32}u_{23} \\
\end{pmatrix}
$
e
$
U = 
\begin{pmatrix}
1 & u_{12}=\frac{a_{12}}{l_{11}} & u_{13}=\frac{a_{13}}{l_{11}}\\
0 & 1 & u_{23}=\frac{a_{23}-l_{21}u_{13}}{l_{22}}\\
0 & 0 & 1
\end{pmatrix}
$
\end{frame}

\begin{frame}
\frametitle{Questão 3.1 pelo método de decomposição LU}
Substituindo os valores da matriz
\[
\begin{pmatrix}
5 & 6 & 12 & | & -1\\
2 & -3 & 0 & | & -1\\
4 & 1 & -1 & | & 6 
\end{pmatrix}
\]

Em

$
L = 
\begin{pmatrix}
\alert<2>{\only<1-2>{a_{11}}}
  \only<3->{5} &
0 &0 \\
%
\alert<2>{\only<1-2>{a_{21}}}
  \only<3->{2} &
\alert<4>{\only<1-4>{a_{22}-l_{21}u_{12}}}
  \only<5>{-3-(2)(1.2)}
  \only<6->{-5.4} &
0 \\
%
\alert<2>{\only<1-2>{a_{31}}}
  \only<3->{4} &
\alert<4>{\only<1-4>{a_{32}-l_{31}u_{12}}}
  \only<5>{1-(4)(1.2)}
  \only<6->{-3.8} &

\alert<8>{\only<1-8>{a_{33} -l_{31}u_{13} -l_{32}u_{23}}}
  \only<9>{-1 -(4)(2.4) -(-3.8)(0.88889)} 
  \only<10->{-7.2222}
\end{pmatrix}
$

e

$
U = 
\begin{pmatrix}
1 & 
\alert<3>{\only<1-3>{\frac{a_{12}}{l_{11}}}}
  \only<4->{1.2} & 
\alert<3>{\only<1-3>{\frac{a_{13}}{l_{11}}}}
  \only<4->{2.4}\\
%
0 & 1 & 
\alert<6>{\only<1-6>{\frac{a_{23}-l_{21}u_{13}}{l_{22}}}}
  \only<7>{\frac{0-(2)(\frac{12}{5})}{-\frac{27}{5}}}
  \only<8->{0.88889} \\
%
0 & 0 & 1
\end{pmatrix}
$
\end{frame}

\begin{frame}
\frametitle{Questão 3.1 pelo método de decomposição LU}
De $Ly=b$, temos:

\[
\begin{pmatrix}
5 & 0 &0 \\
%
2 & -5.4 & 0 \\
%
4 & -3.8 & -7.2222
\end{pmatrix}
\begin{pmatrix}
y_1 \\
y_2 \\
y_3
\end{pmatrix}
=
\begin{pmatrix}
-1\\
-1\\
6
\end{pmatrix}
\]
Portanto:
\[
\left\{
\begin{array}{lll}
5 y_1 &= -1 &\implies y_1 = -0.2\\
2 y_1 -5.4 y_2 &= -1 &\implies y_2 = 0.11111\\
4 y_1 -3.8 y_2 -7.2222 y_3 &= 6 &\implies y_3 = -1
\end{array}
\right.
\]
\end{frame}

\begin{frame}
\frametitle{Questão 3.1 pelo método de decomposição LU}
De $Ux=y$, temos:
\[
\begin{pmatrix}
1 & 1.2 & 2.4\\
0 & 1 & 0.88889 \\
0 & 0 & 1
\end{pmatrix}
%
\begin{pmatrix}
x_1\\
x_2\\
x_3
\end{pmatrix}
%
=
%
\begin{pmatrix}
-0.2\\
0.11111 \\
-1
\end{pmatrix}
\]

Portanto:
\[
\left\{
\begin{array}{rrrlll}
    &          &            &          &        &x_3 = -1 \\
    & x_2      &+0.88889 x_3 &= 0.11111 &\implies&  x_2 = 1\\
x_1 & +1.2 x_2 &+2.4 x_3    &= -0.2    &\implies&  x_1 = 1
\end{array}
\right.
\]
\end{frame}

\begin{frame}
\frametitle{Questão 3.2a}
Determine, aproximadamente, o vetor solução do sistema de equações lineares a seguir, com ``tolerância'' de $10^{-2}$, isto é, faça iterações até que:
\[
\max_{1 \le i \le n}\left| x_i^{(k+1)} - x_i^{(k)} \right| \le 10^{-2}
\]
Caso isso não ocorra até que $k = 2$, pare. Use 3 casas decimais e arredondamento padrão.
%
Parta do vetor nulo e use os métodos iterativos de Jacobi e Gauss-Seidel.
\[
\left\{
\begin{array}{l}
7 x_1 + 6x_2 + 15 x_3 = 4\\
2 x_1 -5x_2 = -8\\
4 x_1 +x_2 - x_3 = 7
\end{array}
\right.
\]
\end{frame}

\begin{frame}
\frametitle{Questão 3.2a por Jacobi}
Nos métodos iterativos transformamos o problema $Ax = b$ em $x = Bx + c$. 
%
No método de Jacobi, aplicamos a fórmula

\[
x_i^{(k+1)} = \frac{b_i - \sum_{j = 1, j \ne i}^{n}a_{ij}x_j^{(k)}}{a_{ii}}
\]

Até que a condição de parada seja satisfeita. 
%
Para uma matriz 3x3, temos:
\[
\left\{
\begin{array}{ll}
x_1^{(k+1)} = &\frac{b_1-a_{12}x_2^{(k)}-a_{13}x_3^{(k)}}{a_{11}}\\
x_2^{(k+1)} = &\frac{b_2-a_{21}x_1^{(k)}-a_{23}x_3^{(k)}}{a_{22}}\\
x_3^{(k+1)} = &\frac{b_3-a_{31}x_1^{(k)}-a_{32}x_2^{(k)}}{a_{33}}
\end{array}
\right.
\]
\end{frame}

\begin{frame}
\frametitle{Questão 3.2a por Jacobi}
\framesubtitle{Convergência}
Para garantir que aplicação do método irá convergir para a solução, precisamos transformar a matriz em uma de diagonal estritamente dominante:
\[
\begin{pmatrix}
\only<1-2>{\alert<2>{7}}\only<3->{\alert<4>{15}} & 
  \only<1->{6} & 
  \only<1-2>{15}\only<3->{7}
  \\
%
\only<1-2>{2}\only<3->{0} & 
  \only<1->{\alert<5>{-5}} & 
  \only<1-2>{0}\only<3->{2}\\
%
\only<1-2>{4}\only<3->{-1} & 
  \only<1->{1 }& 
  \only<1-2>{-1}\only<3->{\alert<6>{4}}
\end{pmatrix}
\]
\only<4>{$|15| > |6|+|7|$ \ok}
\only<5>{$|-5| > |0|+|2|$ \ok}
\only<6>{$|4| > |-1|+|1|$ \ok}

\end{frame}

\begin{frame}
\frametitle{Questão 3.2a por Jacobi}
Aplicando para a matriz
\[
\begin{pmatrix}
15 & 6 & 7 & | & 4\\
0 & -5 & 2 & | & -8\\
-1 & 1 & 4 & | & 7
\end{pmatrix}
\]
Temos:
\[
\left\{
\begin{array}{ll}
x_1^{(k+1)} = &
  \only<1>{\frac{b_1-a_{12}x_2^{(k)}-a_{13}x_3^{(k)}}{a_{11}}}
  \only<2->{\frac{4 -6x_2^{(k)} -7x_3^{(k)}}{15}}\\
%
x_2^{(k+1)} = &
  \only<1>{\frac{b_2-a_{21}x_1^{(k)}-a_{23}x_3^{(k)}}{a_{22}}}
  \only<2>{\frac{-8 -0x_1^{(k)} -2x_3^{(k)}}{-5}}
  \only<3->{\frac{8  +2x_3^{(k)}}{5}}\\
%
x_3^{(k+1)} = &
  \only<1>{\frac{b_3-a_{31}x_1^{(k)}-a_{32}x_2^{(k)}}{a_{33}}}
  \only<2>{\frac{7 +1x_1^{(k)} -1x_2^{(k)}}{4}}
  \only<3->{\frac{7 +x_1^{(k)} -x_2^{(k)}}{4}}
\end{array}
\right.
\]
\end{frame}

\begin{frame}
\frametitle{Questão 3.2a por Jacobi}
De
\[
\left\{
\begin{array}{ll}
x_1^{(k+1)} = &\frac{4 -6x_2^{(k)} -7x_3^{(k)}}{15}\\
%
x_2^{(k+1)} = &\frac{8  +2x_3^{(k)}}{5}\\
%
x_3^{(k+1)} = &\frac{7 +x_1^{(k)} -x_2^{(k)}}{4}
\end{array}
\right.
\]
Montamos a tabela:
\begin{center}
\begin{tabular}{r|r|r|r|c}
$k$ & $x_1^{(k)}$ & $x_2^{(k)}$ & $x_3^{(k)}$ & Parada\\
\hline
\hline
$0$ & $0$ & $0$ & $0$ & --\\
\hline
\only<2->{$1$} & \only<2->{$0.2667$} & \only<2->{$1.6$} & \only<2->{$1.75$} & \only<2->{$1.75 \le 10^{-2} \lor k = 2$}\\
\hline
\only<3->{$2$} & \only<3->{$-1.19$} & \only<3->{$2.3$} & \only<3->{$1.417$} & \only<3->{$1.4567 \le 10^{-2} \lor \alert<4>{k = 2}$}\\
\hline
\end{tabular}
\end{center}
\end{frame}

\begin{frame}
\frametitle{Questão 3.2a por Gauss-Seidel}
No método de Gauss-Seidel, aplicamos a fórmula

\[
x_i^{(k+1)} = \frac{b_i %
  - \sum_{j = 1}^{i-1}a_{ij}x_j^{(k+1)}%
  - \sum_{j = i+1}^{n}a_{ij}x_j^{(k)}%
  }{a_{ii}}
\]

Até que a condição de parada seja satisfeita. 
%
Para uma matriz 3x3, temos:
\[
\left\{
\begin{array}{ll}
x_1^{(k+1)} = &\frac{b_1-a_{12}x_2^{(k)}-a_{13}x_3^{(k)}}{a_{11}}\\
x_2^{(k+1)} = &\frac{b_2-a_{21}x_1^{(k+1)}-a_{23}x_3^{(k)}}{a_{22}}\\
x_3^{(k+1)} = &\frac{b_3-a_{31}x_1^{(k+1)}-a_{32}x_2^{(k+1)}}{a_{33}}
\end{array}
\right.
\]
\end{frame}

\begin{frame}
\frametitle{Questão 3.2a por Gauss-Seidel}
Aplicando para a matriz
\[
\begin{pmatrix}
15 & 6 & 7 & | & 4\\
0 & -5 & 2 & | & -8\\
-1 & 1 & 4 & | & 7
\end{pmatrix}
\]
Temos:
\[
\left\{
\begin{array}{ll}
x_1^{(k+1)} = &
  \only<1>{\frac{b_1-a_{12}x_2^{(k)}-a_{13}x_3^{(k)}}{a_{11}}}
  \only<2->{\frac{4 -6x_2^{(k)} -7x_3^{(k)}}{15}}\\
%
x_2^{(k+1)} = &
  \only<1>{\frac{b_2-a_{21}x_1^{(k+1)}-a_{23}x_3^{(k)}}{a_{22}}}
  \only<2>{\frac{-8 -0x_1^{(k+1)} -2x_3^{(k)}}{-5}}
  \only<3->{\frac{8  +2x_3^{(k)}}{5}}\\
%
x_3^{(k+1)} = &
  \only<1>{\frac{b_3-a_{31}x_1^{(k+1)}-a_{32}x_2^{(k+1)}}{a_{33}}}
  \only<2>{\frac{7 +1x_1^{(k+1)} -1x_2^{(k+1)}}{-1}}
  \only<3->{-7 -x_1^{(k+1)} +x_2^{(k+1)}}
\end{array}
\right.
\]

\end{frame}

\begin{frame}
\frametitle{Questão 3.2a por Gauss-Seidel}
De 
\[
\left\{
\begin{array}{ll}
x_1^{(k+1)} = &\frac{4 -6x_2^{(k)} -7x_3^{(k)}}{15}\\
%
x_2^{(k+1)} = &\frac{8  +2x_3^{(k)}}{5}\\
%
x_3^{(k+1)} = &-7 -x_1^{(k+1)} +x_2^{(k+1)}
\end{array}
\right.
\]

Montamos uma tabela
\begin{center}
\begin{tabular}{r|r|r|r|c}
$k$ & $x_1^{(k)}$ & $x_2^{(k)}$ & $x_3^{(k)}$ & Parada\\
\hline
\hline
$0$ & $0$ & $0$ & $0$ & --\\
\hline
\only<2->{$1$} & \only<2->{$0.2667$} & \only<2->{$1.6$} & \only<2->{$1.417$} & \only<2->{$1.6 \le 10^{-2} \lor k = 2$}\\
\hline
\only<3->{$2$} & \only<3->{$-1.0346$} & \only<3->{$2.1668$} & \only<3->{$0.8775$} & \only<3->{$1.3013 \le 10^{-2} \lor \alert<4>{k = 2}$}\\
\hline
\end{tabular}
\end{center}
\onslide<5>{
Por Gauss-Seidel, $x =
\begin{pmatrix}
-1.0346 & 2.1668 & 0.8775
\end{pmatrix}%
^T$.
Por Jacobi, $x =
\begin{pmatrix}
-1.19 & 2.3 & 1.417
\end{pmatrix}%
^T$.}
\end{frame}


\lecture[Aula 13]{Ajustamento e método dos mínimos quadrados}{aula13}
\section{Introdução}

\begin{frame}
\frametitle{Introdução}

\begin{itemize}
  \item Quando estudamos um fenômeno de forma experimental, é comum termos um conjunto de valores tabelados
  \item Utilizando tais informações podemos levantar várias questões
    \begin{itemize}
      \item Qual a relação existente entre $\x$ e $\f\left(x\right)$?
      \item Qual o valor de $\f\left(x\right)$ para um determinado $\x$ fora do tabelamento?
    \end{itemize}
\end{itemize}
\end{frame}

\begin{frame}
\frametitle{Introdução}

\begin{itemize}
  \item Nestas circustâncias, temos um tabelamento da forma:
%
\begin{center}
\onslide<2->{
\begin{tabular}{|c|c|c|c|c|}
\hline
$\x_i$ & $\x_0$ & $\x_1$ & \ldots & $\x_n$\\
\hline
$\f\left(\x_i\right)$ & $\f\left(\x_0\right)$ & $\f\left(\x_1\right)$ & \ldots & $\f\left(\x_n\right)$\\
\hline
\end{tabular}
}
\end{center}
%
  \item Como podemos usar o tabelamento para calcular o valor da função $\f$ desconhecida em pontos não tabelados?
  \item $\f\left(\x\right)$ mapeia algum fenômeno com dados colhidos de forma experimental
    \begin{itemize}
      \item<3-> Não temos certeza sobre corretude dos dados colhidos
    \end{itemize}
\end{itemize}

\end{frame}

\endlecture

\lecture[Aula 14]{Interpolação -- Lagrange}{aula14}
\section{Interpolação}

\subsection{Introdução}

\begin{frame}{Pergunta}
\begin{itemize}[<+->]
  \item Em um dado problema, foram encontrados os seguintes pontos:
\begin{tabular}{r|rrr}
\bfseries $\x_i$ & \hspace{0.3cm} -1 & \hspace{0.3cm} 1 & \hspace{0.3cm} 2 \\
\hline
\bfseries $\f\left(\x_i\right)$ & -4 & -2 & -10 \\
\end{tabular}
  \item Como poderíamos estimar os valores de $\f\left(\x_i\right)$ para $x=0,5$? E para $1,2$? E $1,6$?
\end{itemize}
\end{frame}

\begin{frame}{O que é interpolação?}
\begin{itemize}
  \item Dado o tabelamento:\\
\begin{tabular}{r|rrrr}
\bfseries $\x_i$ & \hspace{0.3cm} $\x_0$ & \hspace{0.3cm} $\x_1$ & \hspace{0.3cm} \ldots & \hspace{0.3cm} $\x_n$ \\
\hline
\bfseries $\f\left(\x_i\right)$ & $\f\left(\x_0\right)$ & $\f\left(\x_1\right)$ & \ldots & $\f\left(\x_i\right)$ \\
\end{tabular}
  \item Como encontrar o valor de $\f$ para algum $\x$ entre $[x_0; x_n]$?
  \begin{itemize}[<+->]
    \item Se $\x$ pertence à tabela, ok.
    \item E se $\x$ não pertence à tabela?
  \end{itemize}
\end{itemize}
\end{frame}

\begin{frame}{Interpolação}

\begin{itemize}[<+->]
  \item Para resolver essa situação, determinaremos uma função $P$ que passe exatamente nesses pontos tabelados e a usaremos para calcular o valor aproximado de $\f$.
  \item No nosso contexto, vamos aproximar $\f$ por um polinômio.
  \item $P$ terá, portanto a forma:
\[
P\left(\x\right) = a_0 \x^m + a_1 \x^{m-1} + a_2 \x^{m-2} + \ldots + a_{m-1} \x + a_m
\]
\end{itemize}

\end{frame}

\subsubsection{Exemplo 1}

\begin{frame}{Exemplo 1}
\begin{itemize}[<+->]
  \item Dados os pontos da tabelados abaixo, descreva o polinômio que passe exatamente pelos pontos tabelados\\
\begin{tabular}{c|rr}
\bfseries $\x_i$ & \hspace{0.2cm} 1 & \hspace{0.2cm} 2 \\
\hline
\bfseries $\f\left(\x_i\right)$ & 3 & 4 \\
\end{tabular}
  \item Uma reta é um polinômio?
  \item Então podemos aproximar por uma reta\ldots\\
\action<+->{
\begin{minipage}{0.45\textwidth}
\begin{tikzpicture}[global scale=0.7]
\begin{axis}[%
  xmin=-0.5, xmax=2.5, ymin=-0.5, ymax=4.5, axis x line=center, axis y line=center
]
  \addplot plot[mark=*,only marks, xshift=-0.5, yshift=-0.5] file {interpolacao-exemplo1-dados1.txt};
  \addplot plot[mark=none, xshift=-0.5, yshift=-0.5] file {interpolacao-exemplo1-dados2.txt};
  
\end{axis}

\end{tikzpicture}
\end{minipage}
}
\begin{minipage}{0.45\textwidth}
\begin{align*}
\action<+->{\x - \x_0 &= m \left(y - y_0\right)}\\
\action<+->{\x - \x_0 &= \frac{\x_1 - \x_0}{y_1 - y_0} \left(y - y_0\right)}\\
\action<+->{\x - 1 &= \frac{2-1}{4-3} \left(y-3\right)}\\
\action<+->{\x - 1 &= 1 \left(y-3\right)}\\
\action<+->{y &= \x + 2}
\end{align*}
\end{minipage}
\end{itemize}
\end{frame}

\begin{frame}{Exemplo 1}
\begin{itemize}
  \item Há outras formas de aproximar os pontos tabelados por um polinômio\\
\begin{minipage}{0.45\textwidth}
\begin{tikzpicture}[global scale=0.7]
\begin{axis}[%
  xmin=-0.5, xmax=2.5, ymin=-0.5, ymax=4.5, axis x line=center, axis y line=center
]
  \addplot plot[mark=*,only marks, xshift=-0.5, yshift=-0.5] file {interpolacao-exemplo1-dados1.txt};
  \addplot plot[mark=none, xshift=-0.5, yshift=-0.5] file {interpolacao-exemplo1-dados3.txt};
  
\end{axis}

\end{tikzpicture}
\end{minipage}
\begin{minipage}{0.45\textwidth}\begin{tikzpicture}[global scale=0.7]
\begin{axis}[%
  xmin=-0.5, xmax=2.5, ymin=-0.5, ymax=4.5, axis x line=center, axis y line=center
]
  \addplot plot[mark=*,only marks, xshift=-0.5, yshift=-0.5] file {interpolacao-exemplo1-dados1.txt};
  \addplot plot[mark=none, xshift=-0.5, yshift=-0.5] file {interpolacao-exemplo1-dados4.txt};
  
\end{axis}

\end{tikzpicture}
\end{minipage}
  \item Entretanto\ldots
\end{itemize}
\end{frame}

\subsection{Polinômio interpolador}

\begin{frame}{Polinômio interpolador}{Definição 5.1}
\begin{itemize}
  \item Seja uma função $\f$\\ 
\begin{tabular}{r|rrrr}
\bfseries $\x_i$ & \hspace{0.3cm} $\x_0$ & \hspace{0.3cm} $\x_1$ & \hspace{0.3cm} \ldots & \hspace{0.3cm} $\x_n$ \\
\hline
\bfseries $\f\left(\x_i\right)$ & $\f\left(\x_0\right)$ & $\f\left(\x_1\right)$ & \ldots & $\f\left(\x_i\right)$ 
\end{tabular}
\\  
%
$P_n$ é o \emph{polinômio interpolador} de $\f$, relativamente aos pontos $x_0$, $x_1$, \ldots, $x_n$ se, e somente, se:
  \begin{itemize}
    \item $P_n\left(\x\right)$ é de grau não superior a $n$
    \item $P_n\left(\x_i\right) = \f\left(x_i\right)$, $i = 0, 1, 2, \ldots, n$
  \end{itemize}
\end{itemize}
\end{frame}

\begin{frame}{Voltando ao Exemplo 1}
\begin{itemize}
  \item Dados os pontos tabelados abaixo, descreva um polinômio que passe exatamente pelos pontos tabelados.\\
\begin{tabular}{r|rr}
\bfseries $\x_i$ & \hspace{0.2cm} 1 & \hspace{0.2cm} 2\\
\hline
\bfseries $\f\left(\x_i\right)$ & 3 & 4 \\
\end{tabular}
  \item A reta $y = \x + 2$ é o único polinômio interpolador que passa por $(1,3)$ e $(2,4)$.
\end{itemize}
\end{frame}

\subsubsection{Exercício}
\begin{frame}{Exercício}{Exemplo 5.1}
\begin{itemize}
  \item De uma função $\f$ obteve-se a tabela  a seguir. Encontre o polinômio interpolador dela, relativo aos pontos dados.\\
\begin{tabular}{r|rrr}
$\x_i$ & $-1$ & $1$ & $2$\\
\hline
$\f\left(\x_i\right)$ & $-4$ & $-2$ & $-10$
\end{tabular}
\end{itemize}
\end{frame}

\begin{frame}{Solução}{Exemplo 5.1}
\begin{itemize}[<+->]
  \item Neste caso, como $n=2$, temos a expectativa de que o grau de $P$ seja igual a 2. Logo,\\
\action<+->{
\[
P\left(\x\right) = a_0 x^2 + a_1 x + a_2
\]
}
  \item Daí o sistema:
\[
a_0 x^2 + a_1 x + a_2 = \f\left(\x_i\right), i = 0, 1, 2
\]
\begin{minipage}{0.45\textwidth}
\action<+->{
\[
\begin{cases}
a_0 - a_1 + a_2 &= -4 \\
a_0 + a_1 + a_2 &= -2 \\
4 a_0 + 2 a_1 + a_2 &= -10
\end{cases}
\]
}
\end{minipage}
\begin{minipage}{0.45\textwidth}
\action<+->{
$a_0 = -3; a_1 = 1; a_2 = 0$
}

\action<+->{
$P\left(\x\right) = -3 x^2  + x$
}
\end{minipage}
\end{itemize}
\end{frame}

\begin{frame}{Comentários}
Essa forma de aproximar funções (interpolação) só será desejável caso tenhamos certeza sobre a corretude dos valores da tabela, pois, de outra forma, não temos uma boa explicação para as perguntas: Por que a preocupação de passar por pontos duvidosos? Não seria melhor um ajustamento? [livro-texto]
\end{frame}

\begin{frame}{Pergunta?}
\begin{itemize}[<+->]
  \item Qual a diferença entre interpolação e ajustamento?
  \item Ajustamento
  \begin{itemize}[<+->]
    \item Num ajustamento, nós construímos uma curva que se aproxima dos pontos, ou seja, se ajusta aos pontos
    \item Podemos extrapolar a análise para além dos extremos
  \end{itemize}
  \item Interpolação
  \begin{itemize}
    \item Na interpolação, nós construímos uma curva que passa por todos os pontos.
    \item Não podemos extrapolar a análise para além dos extremos.  Ajuda a estimar pontos entre $[x_0; x_n]$
  \end{itemize}
\end{itemize}
\end{frame}

\begin{frame}{Voltando ao Exemplo 5.1}
\begin{itemize}
  \item De uma função $\f$ obteve-se a tabela  a seguir. Encontre o polinômio interpolador dela, relativo aos pontos dados.\\
\begin{tabular}{r|rrr}
$\x_i$ & $-1$ & $1$ & $2$\\
\hline
$\f\left(\x_i\right)$ & $-4$ & $-2$ & $-10$
\end{tabular}
  \item Nós utilizamos sistemas lineares para resolver o problema.
  \item Existe outra forma?
\end{itemize}
\end{frame}

\subsection{Lagrange}

\begin{frame}{Polinômio interpolador de Lagrange}{Teorema 5.1}
\begin{itemize}
  \item Teorema 5.1 (existência e unicidade). Dada uma função $\f$\\
%
\begin{tabular}{r|rrrr}
\bfseries $\x_i$ & \hspace{0.3cm} $\x_0$ & \hspace{0.3cm} $\x_1$ & \hspace{0.3cm} \ldots & \hspace{0.3cm} $\x_n$ \\
\hline
\bfseries $\f\left(\x_i\right)$ & $\f\left(\x_0\right)$ & $\f\left(\x_1\right)$ & \ldots & $\f\left(\x_i\right)$ 
\end{tabular}\\
%
existe um único polinômio $P_n$ interpolador de $\f$, relativamente aos pontos $x_0, x_1, \ldots, x_n$. Ele é dado por:
%
\[
P_n\left(\x\right) = \sum_{i=0}^{n} \f \left(\x_i\right) \lagrange{i}{n}\left(\x\right)
\]
%
onde 
\[
\lagrange{i}{n}\left(\x\right) = \prod_{\substack{j=0\\j\ne i}}^{n} \frac{\x - \x_j}{\x_i - \x_j}
\]
\end{itemize}
\end{frame}

\begin{frame}{Exemplo}
\begin{itemize}
  \item Qual o polinômio interpolador de Lagrange para o Exemplo 5.1?\\
\begin{tabular}{r|rrr}
$\x_i$ & $-1$ & $1$ & $2$\\
\hline
$\f\left(\x_i\right)$ & $-4$ & $-2$ & $-10$
\end{tabular}
\begin{align*}
P_n\left(\x\right) &= \sum_{i=0}^{n} \f \left(\x_i\right) \lagrange{i}{n}\left(\x\right)\\
%
P_2\left(\x\right) &= -4 \lagrange{0}{2} \left(\x\right) -2 \lagrange{1}{2} \left(\x\right) -10 \lagrange{2}{2} \left(x\right)
\end{align*}
  \item Explicitando $\lagrange{i}{2}$: $\lagrange{i}{n}\left(\x\right) = \prod_{\substack{j=0\\j\ne i}}^{n} \frac{\x - \x_j}{\x_i - \x_j}$\\
\begin{minipage}{0.45\textwidth}
\begin{align*}
\action<2->{
\lagrange{0}{2} &= %
\frac{%
  \only<2-3>{
  \left(\x - \only<2>{\x_1}\only<3>{1}\right)%
  \left(\x - \only<2>{\x_2}\only<3>{2}\right)
  }
  \only<4->{x^2 - 3 x + 2}%
  }%
  {%
  \only<2-3>{
  \left(\only<2>{\x_0}\only<3>{-1} - \only<2>{\x_1}\only<3>{1}\right)%
  \left(\only<2>{\x_0}\only<3>{-1} - \only<2>{\x_2}\only<3>{2}\right)%
  }
  \only<4->{6}
  }
}\\
\action<5->{
\lagrange{1}{2} &= %
\frac{%
  \only<5-6>{
  \left(\x - \only<5>{\x_0}\only<6>{(-1)}\right)%
  \left(\x - \only<5>{\x_2}\only<6>{2}\right)
  }
  \only<7->{-x^2 + x + 2}%
  }%
  {%
  \only<5-6>{
  \left(\only<5>{\x_1}\only<6>{1} - \only<5>{\x_0}\only<6>{(-1)}\right)%
  \left(\only<5>{\x_1}\only<6>{1} - \only<5>{\x_2}\only<6>{2}\right)%
  }
  \only<7->{2}
  }
}
\end{align*}
\end{minipage}
\begin{minipage}{0.45\textwidth}
\begin{align*}
\action<8->{
\lagrange{2}{2} &= %
\frac{%
  \only<8-9>{
  \left(\x - \only<8>{\x_0}\only<9>{(-1)}\right)%
  \left(\x - \only<8>{\x_1}\only<9>{1}\right)
  }
  \only<10->{x^2 -1}%
  }%
  {%
  \only<8-9>{
  \left(\only<8>{\x_2}\only<9>{2} - \only<8>{\x_0}\only<9>{(-1)}\right)%
  \left(\only<8>{\x_2}\only<9>{2} - \only<8>{\x_1}\only<9>{1}\right)%
  }
  \only<10->{3}
  }
}
\end{align*}
\end{minipage}

\end{itemize}
\end{frame}

\begin{frame}{Substituindo em $P_2$}
\begin{itemize}
  \item De: 
\begin{align*}
P_2\left(\x\right) &= -4 \lagrange{0}{2} \left(\x\right) -2 \lagrange{1}{2} \left(\x\right) -10 \lagrange{2}{2} \left(x\right)\\
\lagrange{0}{2} &= \frac{\x^2 - 3 \x + 2}{6}\\
\lagrange{1}{2} &= \frac{-\x^2 + \x +2}{2}\\
\lagrange{2}{2} &= \frac{\x^2 -1}{3}
\end{align*}

\begin{align*}
P_2\left(\x\right) &= -4 \left(\frac{\x^2 - 3 \x + 2}{6}\right)  -2 \left(\frac{-\x^2 + \x +2}{2}\right) -10 \left(\frac{\x^2 -1}{3}\right)\\
P_2\left(\x\right) &= -3 \x^2 + \x
\end{align*}

\end{itemize}

\end{frame}

\begin{frame}{Exercício 5.2}
\begin{enumerate}
\setcounter{enumi}{1}
  \item Seja a função $\f$ dada pela tabela\\
\begin{tabular}{r|rrr}
$\x_i$ & \hspace{0.3cm} $-1$ & \hspace{0.3cm} $0$ & \hspace{0.3cm} $1$\\
\hline
$\f \left(\x_i\right)$ & $-3$ & $-1$ & $1$ 
\end{tabular}
  \begin{enumerate}
    \item\label{item:exe52-1} Determine  o polinômio interpolador de $\f$ relativamente à tabela dada
    \item\label{item:exe52-2} Verifique que $g \left(\x\right) = 2 x^3 -1$ satisfaz a condição $g\left(\x_i\right) = \f \left(\x_i\right)$, $i = 0, 1, 2$
    \item\label{item:exe52-3} O que você pode afirmar a respeito do item b? Justifique.
  \end{enumerate}
\end{enumerate}
\end{frame}

\begin{frame}{Exercício 5.2}{\Cref{item:exe52-1}}
\begin{itemize}
  \item Para os pontos tabelados\\
\begin{tabular}{r|rrr}
$\x_i$ & \hspace{0.3cm} $-1$ & \hspace{0.3cm} $0$ & \hspace{0.3cm} $1$\\
\hline
$\f \left(\x_i\right)$ & $-3$ & $-1$ & $1$ 
\end{tabular}
  \item Usando a fórmula do polinômio interpolador:
\begin{align*}
P_n\left(\x\right) &= \sum_{i=0}^{n} \f \left(\x_i\right) \lagrange{i}{n}\left(\x\right)\\
%
P_2\left(\x\right) &= -3 \lagrange{0}{2} \left(\x\right) -1 \lagrange{1}{2} \left(\x\right) +1 \lagrange{2}{2} \left(x\right)
\end{align*}
\begin{minipage}{0.45\textwidth}
\begin{align*}
\action<2->{
\lagrange{0}{2} &= %
\frac{%
  \only<2-3>{
  \left(\x - \only<2>{\x_1}\only<3>{0}\right)%
  \left(\x - \only<2>{\x_2}\only<3>{1}\right)
  }
  \only<4->{x^2 - x}%
  }%
  {%
  \only<2-3>{
  \left(\only<2>{\x_0}\only<3>{-1} - \only<2>{\x_1}\only<3>{0}\right)%
  \left(\only<2>{\x_0}\only<3>{-1} - \only<2>{\x_2}\only<3>{1}\right)%
  }
  \only<4->{2}
  }
}\\
\action<5->{
\lagrange{1}{2} &= %
\frac{%
  \only<5-6>{
  \left(\x - \only<5>{\x_0}\only<6>{(-1)}\right)%
  \left(\x - \only<5>{\x_2}\only<6>{1}\right)
  }
  \only<7->{x^2 -1}%
  }%
  {%
  \only<5-6>{
  \left(\only<5>{\x_1}\only<6>{0} - \only<5>{\x_0}\only<6>{(-1)}\right)%
  \left(\only<5>{\x_1}\only<6>{0} - \only<5>{\x_2}\only<6>{1}\right)%
  }
  \only<7->{-1}
  }
}
\end{align*}
\end{minipage}
\begin{minipage}{0.45\textwidth}
\begin{align*}
\action<8->{
\lagrange{2}{2} &= %
\frac{%
  \only<8-9>{
  \left(\x - \only<8>{\x_0}\only<9>{(-1)}\right)%
  \left(\x - \only<8>{\x_1}\only<9>{0}\right)
  }
  \only<10->{x^2 +x}%
  }%
  {%
  \only<8-9>{
  \left(\only<8>{\x_2}\only<9>{1} - \only<8>{\x_0}\only<9>{(-1)}\right)%
  \left(\only<8>{\x_2}\only<9>{1} - \only<8>{\x_1}\only<9>{0}\right)%
  }
  \only<10->{2}
  }
}
\end{align*}
\end{minipage}
\end{itemize}
\end{frame}

\begin{frame}{Exercício 5.2}{\Cref{item:exe52-1}}
\begin{itemize}
  \item Fazendo as substituições em $P_2$:
\begin{align*}
P_2\left(\x\right) &= -3 \lagrange{0}{2} \left(\x\right) -1 \lagrange{1}{2} \left(\x\right) +1 \lagrange{2}{2} \left(x\right)\\
\lagrange{0}{2} &= \frac{\x^2 - \x}{2}\\
\lagrange{1}{2} &= -\x^2 +1\\
\lagrange{2}{2} &= \frac{\x^2 +x}{2}
\end{align*}

\begin{align*}
P_2\left(\x\right) &= 
  -3 \only<1>{\lagrange{0}{2}}\only<2->{\left(\frac{\x^2 - \x}{2}\right)}%
  -1 \only<1-2>{\lagrange{1}{2}}\only<3->{\left(-\x^2 +1\right)}%
  +1 \only<1-3>{\lagrange{2}{2}}\only<4->{\left(\frac{\x^2 +x}{2}\right)}\\
\onslide<5->{P_2\left(\x\right) &= 2x -1}
\end{align*}
\end{itemize}
\end{frame}

\begin{frame}{Exercício 5.2}{\Cref{item:exe52-2,item:exe52-3}}
\Cref{item:exe52-2}: Verifique que $g \left(\x\right) = 2 x^3 -1$ satisfaz a condição $g\left(\x_i\right) = \f \left(\x_i\right)$, $i = 0, 1, 2$
\begin{itemize}
  \item<2-> Sim, satisfaz.
\end{itemize}
\Cref{item:exe52-3}: O que você pode afirmar a respeito do item b? Justifique.
\begin{itemize}
  \item<3-> Apesar de $g\left(x\right)$ satisfazer o tabelamento, não é um polinômio interpolador, pois seu grau é maior que 2.
\end{itemize}
\end{frame}
\endlecture

%\begin{frame}
%\frametitle{Outline}
%\tableofcontents
% You might wish to add the option [pausesections]
%\end{frame}

\end{document}
