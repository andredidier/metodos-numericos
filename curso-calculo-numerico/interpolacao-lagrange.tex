\section{Interpolação}

\subsection{Introdução}

\begin{frame}{Pergunta}
\begin{itemize}[<+->]
  \item Em um dado problema, foram encontrados os seguintes pontos:
\begin{tabular}{r|rrr}
\bfseries $\x_i$ & \hspace{0.3cm} -1 & \hspace{0.3cm} 1 & \hspace{0.3cm} 2 \\
\hline
\bfseries $\f\left(\x_i\right)$ & -4 & -2 & -10 \\
\end{tabular}
  \item Como poderíamos estimar os valores de $\f\left(\x_i\right)$ para $x=0,5$? E para $1,2$? E $1,6$?
\end{itemize}
\end{frame}

\begin{frame}{O que é interpolação?}
\begin{itemize}
  \item Dado o tabelamento:\\
\begin{tabular}{r|rrrr}
\bfseries $\x_i$ & \hspace{0.3cm} $\x_0$ & \hspace{0.3cm} $\x_1$ & \hspace{0.3cm} \ldots & \hspace{0.3cm} $\x_n$ \\
\hline
\bfseries $\f\left(\x_i\right)$ & $\f\left(\x_0\right)$ & $\f\left(\x_1\right)$ & \ldots & $\f\left(\x_i\right)$ \\
\end{tabular}
  \item Como encontrar o valor de $\f$ para algum $\x$ entre $[x_0; x_n]$?
  \begin{itemize}[<+->]
    \item Se $\x$ pertence à tabela, ok.
    \item E se $\x$ não pertence à tabela?
  \end{itemize}
\end{itemize}
\end{frame}

\begin{frame}{Interpolação}

\begin{itemize}[<+->]
  \item Para resolver essa situação, determinaremos uma função $P$ que passe exatamente nesses pontos tabelados e a usaremos para calcular o valor aproximado de $\f$.
  \item No nosso contexto, vamos aproximar $\f$ por um polinômio.
  \item $P$ terá, portanto, a forma:
\[
P\left(\x\right) = a_0 \x^m + a_1 \x^{m-1} + a_2 \x^{m-2} + \ldots + a_{m-1} \x + a_m
\]
\end{itemize}

\end{frame}

\subsubsection{Exemplo 1}

\begin{frame}{Exemplo 1}
\begin{itemize}[<+->]
  \item Dados os pontos da tabelados abaixo, descreva o polinômio que passe exatamente pelos pontos tabelados\\
\begin{tabular}{c|rr}
\bfseries $\x_i$ & \hspace{0.2cm} 1 & \hspace{0.2cm} 2 \\
\hline
\bfseries $\f\left(\x_i\right)$ & 3 & 4 \\
\end{tabular}
  \item Uma reta é um polinômio?
  \item Então podemos aproximar por uma reta\ldots\\
\action<+->{
\begin{minipage}{0.45\textwidth}
\begin{tikzpicture}[global scale=0.7]
\begin{axis}[%
  xmin=-0.5, xmax=2.5, ymin=-0.5, ymax=4.5, axis x line=center, axis y line=center
]
  \addplot plot[mark=*,only marks, xshift=-0.5, yshift=-0.5] file {interpolacao-exemplo1-dados1.txt};
  \addplot plot[mark=none, xshift=-0.5, yshift=-0.5] file {interpolacao-exemplo1-dados2.txt};
  
\end{axis}

\end{tikzpicture}
\end{minipage}
}
\begin{minipage}{0.45\textwidth}
\begin{align*}
\action<+->{\x - \x_0 &= m \left(y - y_0\right)}\\
\action<+->{\x - \x_0 &= \frac{\x_1 - \x_0}{y_1 - y_0} \left(y - y_0\right)}\\
\action<+->{\x - 1 &= \frac{2-1}{4-3} \left(y-3\right)}\\
\action<+->{\x - 1 &= 1 \left(y-3\right)}\\
\action<+->{y &= \x + 2}
\end{align*}
\end{minipage}
\end{itemize}
\end{frame}

\begin{frame}{Exemplo 1}
\begin{itemize}
  \item Há outras formas de aproximar os pontos tabelados por um polinômio\\
\begin{minipage}{0.45\textwidth}
\begin{tikzpicture}[global scale=0.7]
\begin{axis}[%
  xmin=-0.5, xmax=2.5, ymin=-0.5, ymax=4.5, axis x line=center, axis y line=center
]
  \addplot plot[mark=*,only marks, xshift=-0.5, yshift=-0.5] file {interpolacao-exemplo1-dados1.txt};
  \addplot plot[mark=none, xshift=-0.5, yshift=-0.5] file {interpolacao-exemplo1-dados3.txt};
  
\end{axis}

\end{tikzpicture}
\end{minipage}
\begin{minipage}{0.45\textwidth}\begin{tikzpicture}[global scale=0.7]
\begin{axis}[%
  xmin=-0.5, xmax=2.5, ymin=-0.5, ymax=4.5, axis x line=center, axis y line=center
]
  \addplot plot[mark=*,only marks, xshift=-0.5, yshift=-0.5] file {interpolacao-exemplo1-dados1.txt};
  \addplot plot[mark=none, xshift=-0.5, yshift=-0.5] file {interpolacao-exemplo1-dados4.txt};
  
\end{axis}

\end{tikzpicture}
\end{minipage}
  \item Entretanto\ldots
\end{itemize}
\end{frame}

\subsection{Polinômio interpolador}

\begin{frame}{Polinômio interpolador}{Definição 5.1}
\begin{itemize}
  \item Seja uma função $\f$\\ 
\begin{tabular}{r|rrrr}
\bfseries $\x_i$ & \hspace{0.3cm} $\x_0$ & \hspace{0.3cm} $\x_1$ & \hspace{0.3cm} \ldots & \hspace{0.3cm} $\x_n$ \\
\hline
\bfseries $\f\left(\x_i\right)$ & $\f\left(\x_0\right)$ & $\f\left(\x_1\right)$ & \ldots & $\f\left(\x_i\right)$ 
\end{tabular}
\\  
%
$P_n$ é o \emph{polinômio interpolador} de $\f$, relativamente aos pontos $x_0$, $x_1$, \ldots, $x_n$ se, e somente, se:
  \begin{itemize}
    \item $P_n\left(\x\right)$ é de grau não superior a $n$
    \item $P_n\left(\x_i\right) = \f\left(x_i\right)$, $i = 0, 1, 2, \ldots, n$
  \end{itemize}
\end{itemize}
\end{frame}

\begin{frame}{Voltando ao Exemplo 1}
\begin{itemize}
  \item Dados os pontos tabelados abaixo, descreva um polinômio que passe exatamente pelos pontos tabelados.\\
\begin{tabular}{r|rr}
\bfseries $\x_i$ & \hspace{0.2cm} 1 & \hspace{0.2cm} 2\\
\hline
\bfseries $\f\left(\x_i\right)$ & 3 & 4 \\
\end{tabular}
  \item A reta $y = \x + 2$ é o único polinômio interpolador que passa por $(1,3)$ e $(2,4)$.
\end{itemize}
\end{frame}

\subsubsection{Exercício}
\begin{frame}{Exercício}{Exemplo 5.1}
\begin{itemize}
  \item De uma função $\f$ obteve-se a tabela  a seguir. Encontre o polinômio interpolador dela, relativo aos pontos dados.\\
\begin{tabular}{r|rrr}
$\x_i$ & $-1$ & $1$ & $2$\\
\hline
$\f\left(\x_i\right)$ & $-4$ & $-2$ & $-10$
\end{tabular}
\end{itemize}
\end{frame}

\begin{frame}{Solução}{Exemplo 5.1}
\begin{itemize}[<+->]
  \item Neste caso, como $n=2$, temos a expectativa de que o grau de $P$ seja igual a 2. Logo,\\
\action<+->{
\[
P\left(\x\right) = a_0 x^2 + a_1 x + a_2
\]
}
  \item Daí o sistema:
\[
a_0 x^2 + a_1 x + a_2 = \f\left(\x_i\right), i = 0, 1, 2
\]
\begin{minipage}{0.45\textwidth}
\action<+->{
\[
\begin{cases}
a_0 - a_1 + a_2 &= -4 \\
a_0 + a_1 + a_2 &= -2 \\
4 a_0 + 2 a_1 + a_2 &= -10
\end{cases}
\]
}
\end{minipage}
\begin{minipage}{0.45\textwidth}
\action<+->{
$a_0 = -3; a_1 = 1; a_2 = 0$
}

\action<+->{
$P\left(\x\right) = -3 x^2  + x$
}
\end{minipage}
\end{itemize}
\end{frame}

\begin{frame}{Comentários}
Essa forma de aproximar funções (interpolação) só será desejável caso tenhamos certeza sobre a corretude dos valores da tabela, pois, de outra forma, não temos uma boa explicação para as perguntas: Por que a preocupação de passar por pontos duvidosos? Não seria melhor um ajustamento? [livro-texto]
\end{frame}

\begin{frame}{Pergunta?}
\begin{itemize}[<+->]
  \item Qual a diferença entre interpolação e ajustamento?
  \item Ajustamento
  \begin{itemize}[<+->]
    \item Num ajustamento, nós construímos uma curva que se aproxima dos pontos, ou seja, se ajusta aos pontos
    \item Podemos extrapolar a análise para além dos extremos
  \end{itemize}
  \item Interpolação
  \begin{itemize}
    \item Na interpolação, nós construímos uma curva que passa por todos os pontos.
    \item Não podemos extrapolar a análise para além dos extremos.  Ajuda a estimar pontos entre $[x_0; x_n]$
  \end{itemize}
\end{itemize}
\end{frame}

\begin{frame}{Voltando ao Exemplo 5.1}
\begin{itemize}
  \item De uma função $\f$ obteve-se a tabela  a seguir. Encontre o polinômio interpolador dela, relativo aos pontos dados.\\
\begin{tabular}{r|rrr}
$\x_i$ & $-1$ & $1$ & $2$\\
\hline
$\f\left(\x_i\right)$ & $-4$ & $-2$ & $-10$
\end{tabular}
  \item Nós utilizamos sistemas lineares para resolver o problema.
  \item Existe outra forma?
\end{itemize}
\end{frame}

\subsection{Lagrange}

\begin{frame}{Polinômio interpolador de Lagrange}{Teorema 5.1}
\begin{itemize}
  \item Teorema 5.1 (existência e unicidade). Dada uma função $\f$\\
%
\begin{tabular}{r|rrrr}
\bfseries $\x_i$ & \hspace{0.3cm} $\x_0$ & \hspace{0.3cm} $\x_1$ & \hspace{0.3cm} \ldots & \hspace{0.3cm} $\x_n$ \\
\hline
\bfseries $\f\left(\x_i\right)$ & $\f\left(\x_0\right)$ & $\f\left(\x_1\right)$ & \ldots & $\f\left(\x_i\right)$ 
\end{tabular}\\
%
existe um único polinômio $P_n$ interpolador de $\f$, relativamente aos pontos $x_0, x_1, \ldots, x_n$. Ele é dado por:
%
\[
P_n\left(\x\right) = \sum_{i=0}^{n} \f \left(\x_i\right) \lagrange{i}{n}\left(\x\right)
\]
%
onde 
\[
\lagrange{i}{n}\left(\x\right) = \prod_{\substack{j=0\\j\ne i}}^{n} \frac{\x - \x_j}{\x_i - \x_j}
\]
\end{itemize}
\end{frame}

\begin{frame}{Exemplo}
\begin{itemize}
  \item Qual o polinômio interpolador de Lagrange para o Exemplo 5.1?\\
\begin{tabular}{r|rrr}
$\x_i$ & $-1$ & $1$ & $2$\\
\hline
$\f\left(\x_i\right)$ & $-4$ & $-2$ & $-10$
\end{tabular}
\begin{align*}
P_n\left(\x\right) &= \sum_{i=0}^{n} \f \left(\x_i\right) \lagrange{i}{n}\left(\x\right)\\
%
P_2\left(\x\right) &= -4 \lagrange{0}{2} \left(\x\right) -2 \lagrange{1}{2} \left(\x\right) -10 \lagrange{2}{2} \left(x\right)
\end{align*}
  \item Explicitando $\lagrange{i}{2}$: $\lagrange{i}{n}\left(\x\right) = \prod_{\substack{j=0\\j\ne i}}^{n} \frac{\x - \x_j}{\x_i - \x_j}$\\
\begin{minipage}{0.45\textwidth}
\begin{align*}
\action<2->{
\lagrange{0}{2} &= %
\frac{%
  \only<2-3>{
  \left(\x - \only<2>{\x_1}\only<3>{1}\right)%
  \left(\x - \only<2>{\x_2}\only<3>{2}\right)
  }
  \only<4->{x^2 - 3 x + 2}%
  }%
  {%
  \only<2-3>{
  \left(\only<2>{\x_0}\only<3>{-1} - \only<2>{\x_1}\only<3>{1}\right)%
  \left(\only<2>{\x_0}\only<3>{-1} - \only<2>{\x_2}\only<3>{2}\right)%
  }
  \only<4->{6}
  }
}\\
\action<5->{
\lagrange{1}{2} &= %
\frac{%
  \only<5-6>{
  \left(\x - \only<5>{\x_0}\only<6>{(-1)}\right)%
  \left(\x - \only<5>{\x_2}\only<6>{2}\right)
  }
  \only<7->{-x^2 + x + 2}%
  }%
  {%
  \only<5-6>{
  \left(\only<5>{\x_1}\only<6>{1} - \only<5>{\x_0}\only<6>{(-1)}\right)%
  \left(\only<5>{\x_1}\only<6>{1} - \only<5>{\x_2}\only<6>{2}\right)%
  }
  \only<7->{2}
  }
}
\end{align*}
\end{minipage}
\begin{minipage}{0.45\textwidth}
\begin{align*}
\action<8->{
\lagrange{2}{2} &= %
\frac{%
  \only<8-9>{
  \left(\x - \only<8>{\x_0}\only<9>{(-1)}\right)%
  \left(\x - \only<8>{\x_1}\only<9>{1}\right)
  }
  \only<10->{x^2 -1}%
  }%
  {%
  \only<8-9>{
  \left(\only<8>{\x_2}\only<9>{2} - \only<8>{\x_0}\only<9>{(-1)}\right)%
  \left(\only<8>{\x_2}\only<9>{2} - \only<8>{\x_1}\only<9>{1}\right)%
  }
  \only<10->{3}
  }
}
\end{align*}
\end{minipage}

\end{itemize}
\end{frame}

\begin{frame}{Substituindo em $P_2$}
\begin{itemize}
  \item De: 
\begin{align*}
P_2\left(\x\right) &= -4 \lagrange{0}{2} \left(\x\right) -2 \lagrange{1}{2} \left(\x\right) -10 \lagrange{2}{2} \left(x\right)\\
\lagrange{0}{2} &= \frac{\x^2 - 3 \x + 2}{6}\\
\lagrange{1}{2} &= \frac{-\x^2 + \x +2}{2}\\
\lagrange{2}{2} &= \frac{\x^2 -1}{3}
\end{align*}

\begin{align*}
P_2\left(\x\right) &= -4 \left(\frac{\x^2 - 3 \x + 2}{6}\right)  -2 \left(\frac{-\x^2 + \x +2}{2}\right) -10 \left(\frac{\x^2 -1}{3}\right)\\
P_2\left(\x\right) &= -3 \x^2 + \x
\end{align*}

\end{itemize}

\end{frame}

\begin{frame}{Exercício 5.2}
\begin{enumerate}
\setcounter{enumi}{1}
  \item Seja a função $\f$ dada pela tabela\\
\begin{tabular}{r|rrr}
$\x_i$ & \hspace{0.3cm} $-1$ & \hspace{0.3cm} $0$ & \hspace{0.3cm} $1$\\
\hline
$\f \left(\x_i\right)$ & $-3$ & $-1$ & $1$ 
\end{tabular}
  \begin{enumerate}
    \item\label{item:exe52-1} Determine  o polinômio interpolador de $\f$ relativamente à tabela dada
    \item\label{item:exe52-2} Verifique que $g \left(\x\right) = 2 x^3 -1$ satisfaz a condição $g\left(\x_i\right) = \f \left(\x_i\right)$, $i = 0, 1, 2$
    \item\label{item:exe52-3} O que você pode afirmar a respeito do item b? Justifique.
  \end{enumerate}
\end{enumerate}
\end{frame}

\begin{frame}{Exercício 5.2}{\Cref{item:exe52-1}}
\begin{itemize}
  \item Para os pontos tabelados\\
\begin{tabular}{r|rrr}
$\x_i$ & \hspace{0.3cm} $-1$ & \hspace{0.3cm} $0$ & \hspace{0.3cm} $1$\\
\hline
$\f \left(\x_i\right)$ & $-3$ & $-1$ & $1$ 
\end{tabular}
  \item Usando a fórmula do polinômio interpolador:
\begin{align*}
P_n\left(\x\right) &= \sum_{i=0}^{n} \f \left(\x_i\right) \lagrange{i}{n}\left(\x\right)\\
%
P_2\left(\x\right) &= -3 \lagrange{0}{2} \left(\x\right) -1 \lagrange{1}{2} \left(\x\right) +1 \lagrange{2}{2} \left(x\right)
\end{align*}
\begin{minipage}{0.45\textwidth}
\begin{align*}
\action<2->{
\lagrange{0}{2} &= %
\frac{%
  \only<2-3>{
  \left(\x - \only<2>{\x_1}\only<3>{0}\right)%
  \left(\x - \only<2>{\x_2}\only<3>{1}\right)
  }
  \only<4->{x^2 - x}%
  }%
  {%
  \only<2-3>{
  \left(\only<2>{\x_0}\only<3>{-1} - \only<2>{\x_1}\only<3>{0}\right)%
  \left(\only<2>{\x_0}\only<3>{-1} - \only<2>{\x_2}\only<3>{1}\right)%
  }
  \only<4->{2}
  }
}\\
\action<5->{
\lagrange{1}{2} &= %
\frac{%
  \only<5-6>{
  \left(\x - \only<5>{\x_0}\only<6>{(-1)}\right)%
  \left(\x - \only<5>{\x_2}\only<6>{1}\right)
  }
  \only<7->{x^2 -1}%
  }%
  {%
  \only<5-6>{
  \left(\only<5>{\x_1}\only<6>{0} - \only<5>{\x_0}\only<6>{(-1)}\right)%
  \left(\only<5>{\x_1}\only<6>{0} - \only<5>{\x_2}\only<6>{1}\right)%
  }
  \only<7->{-1}
  }
}
\end{align*}
\end{minipage}
\begin{minipage}{0.45\textwidth}
\begin{align*}
\action<8->{
\lagrange{2}{2} &= %
\frac{%
  \only<8-9>{
  \left(\x - \only<8>{\x_0}\only<9>{(-1)}\right)%
  \left(\x - \only<8>{\x_1}\only<9>{0}\right)
  }
  \only<10->{x^2 +x}%
  }%
  {%
  \only<8-9>{
  \left(\only<8>{\x_2}\only<9>{1} - \only<8>{\x_0}\only<9>{(-1)}\right)%
  \left(\only<8>{\x_2}\only<9>{1} - \only<8>{\x_1}\only<9>{0}\right)%
  }
  \only<10->{2}
  }
}
\end{align*}
\end{minipage}
\end{itemize}
\end{frame}

\begin{frame}{Exercício 5.2}{\Cref{item:exe52-1}}
\begin{itemize}
  \item Fazendo as substituições em $P_2$:
\begin{align*}
P_2\left(\x\right) &= -3 \lagrange{0}{2} \left(\x\right) -1 \lagrange{1}{2} \left(\x\right) +1 \lagrange{2}{2} \left(x\right)\\
\lagrange{0}{2} &= \frac{\x^2 - \x}{2}\\
\lagrange{1}{2} &= -\x^2 +1\\
\lagrange{2}{2} &= \frac{\x^2 +x}{2}
\end{align*}

\begin{align*}
P_2\left(\x\right) &= 
  -3 \only<1>{\lagrange{0}{2}}\only<2->{\left(\frac{\x^2 - \x}{2}\right)}%
  -1 \only<1-2>{\lagrange{1}{2}}\only<3->{\left(-\x^2 +1\right)}%
  +1 \only<1-3>{\lagrange{2}{2}}\only<4->{\left(\frac{\x^2 +x}{2}\right)}\\
\onslide<5->{P_2\left(\x\right) &= 2x -1}
\end{align*}
\end{itemize}
\end{frame}

\begin{frame}{Exercício 5.2}{\Cref{item:exe52-2,item:exe52-3}}
\Cref{item:exe52-2}: Verifique que $g \left(\x\right) = 2 x^3 -1$ satisfaz a condição $g\left(\x_i\right) = \f \left(\x_i\right)$, $i = 0, 1, 2$
\begin{itemize}
  \item<2-> Sim, satisfaz.
\end{itemize}
\Cref{item:exe52-3}: O que você pode afirmar a respeito do item b? Justifique.
\begin{itemize}
  \item<3-> Apesar de $g\left(x\right)$ satisfazer o tabelamento, não é um polinômio interpolador, pois seu grau é maior que 2.
\end{itemize}
\end{frame}